\subsection{凝胶的制备、性质和结构}
凝胶是一种具有半固态、富液体、高粘滞性且含有微孔的二组分体系,其中一个组分的分子键合成三维网络,而另一组分通过渗透而形成连续相。凝胶介质不仅是指由硅酸钠溶液制备的硅水凝胶,而且还包括各种有机凝胶,例如从海藻中得到的碳水化合物高聚物(琼脂\footnote{原文作“酯”。}胶)、类似于蛋白质结构的明胶、聚丙烯酰胺、醋酸纤维、四甲氧基硅、四乙氧基硅、硬脂酸盐、铝酸盐等。但硅水凝胶是最普遍使用的一种凝胶,这是由于它比其他有机凝胶更适用于绝大多数晶体的生长。

将硅酸钠($\rm Na_2SiO_3\cdot 9H_2O$)加入水中,酸化后可有硅酸产生,硅酸的一个重要特性是它的聚合作用,影响硅酸聚合作用的因素较多,其中最重要的因素是溶液的酸度。胶凝时间与溶液的pH值的关系曲线如图3.40所示。

(图3.40)

欲得某一确定pH值的凝胶介质,在不断搅拌情况下,准确地加入需要量的酸于硅酸钠水溶液中,该溶液的酸度值决定着凝胶的聚合过程和速度。胶凝时间可从几秒钟、几分钟到几天、几十天或更长的时间。

当硅酸钠在强酸溶液中时,硅酸离子的逐步酸化过程为

(化学图)

氢离子与硅酸中的氧原子结合后,硅氧(!!!化学图!!!)键长即增加。若配位在Si原子周围的$\rm Si-O$键键长都增加,则其配位数即可增大。因此至少在$\rm H_3SiO_4^+$离子(式IV)内的Si的配位数可增到6,可写为

(化学图)

在碱性溶液中,硅酸存在的形式为式(I)和式(II),两者均带负电荷,顾客认为不易起聚合作用或聚合作用极慢,且随碱度的增加而聚合减慢。

在微碱性、中性和微酸性溶液中,硅酸存在的主要形式为式(II)和式(III),两者相遇便可发生聚合作用,从而生成

(化学图)

此种聚合将以$\rm Si-O$键为基础在三维空间延续而形成网络结构。

在较高酸度的溶液中,离子(III)与(IV)相互作用,聚合成为

(化学图)

以上两种类型的聚合反应机制,前者有$\rm OH^-$离子释出,故在聚合过程中溶液的pH值增高,而后者会使溶液的pH值略有降低。当两种反应同时进行时,溶液的pH值便无明显的\footnote{原文作“地”。}变化。

采用酒石酸、醋酸或硝酸作为胶凝剂所制得的凝胶均为透明的。但弱酸作为胶凝剂应用的比较多,这首先是由于用弱酸作胶凝剂,凝胶的pH值随时间变化较小;其二是由于无机酸或多或少地对晶体生长具有一定的损害作用。但使用何种酸为宜,则要视所生长晶体的类型而定。

将酸加入硅酸钠溶液中时,须严防凝胶中气泡的形成,若气泡一旦形成,将很难把气泡从凝胶中排除。此乃往往会阻止晶体生长。

凝胶液的pH值是决定凝胶结构及其性质非常重要的因素,同时在晶体生长过程中也起着重要的作用。硅水凝胶的结构网络,具有两种类型的微孔,一类微孔接近分子尺度,另一类表现为正常的毛细管大小。硅水凝胶的X射线衍射花样同硅玻璃相比,凝胶所具有的均匀性差些。在温和的条件下对硅水凝胶进行干燥,凝胶的总体积变化甚小,干燥过的凝胶易于作X射线衍射实验,可获得更多的结构信息,在对比的情况下,对硅水凝胶也可获得很有意义的实验结果。经估算,硅水凝胶的有效微孔直径约为$5-10\rm\ nm$,脱水收缩作用对微孔平均尺寸的影响甚小。此外,在制备凝胶时所采用的酸的性质和浓度,硅酸钠溶液的密度及硅酸钠的纯度,环境温度,化学试剂在凝胶中的浓度等都是影响凝胶法晶体生长的重要因素。

