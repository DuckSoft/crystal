\subsection{分配系数}
溶液生长中平衡的一个重要方面是除溶质和溶剂外的第三个组分(包括杂质、掺质和同位素置换等)在固相中的溶解度问题,也就是该组分在固相和与之相平衡的液相之间的分配问题。

如果第三个组分(以下简称物质)在溶液和固相之间达成平衡时,有
\begin{equation}
\text{物质在液态溶液中的活度}(a_l) \rightleftharpoons \text{物质在固态溶液中的活度}(a_s),
\end{equation}
则
\begin{equation}
k_0=a_s(e)/a_l(e),
\end{equation}
$k_0$称为平衡分配系数。对于接近理想溶液的稀溶液来说,有
\begin{equation}
k_0=c_s(e)/c_l(e),
\end{equation}
$c_s(e)$和$c_l(e)$分别是物质在固相和溶液中的平衡浓度,即在液相和固相分界处的浓度。显然,要达到平衡时的分配系数,就要求晶体在溶液中既不溶解也不生长,或是生长速度很慢以至接近于平衡状态。另外,物质在两相中的浓度也不宜太大,以便用浓度来代替活度。

晶体在溶液中生长时,靠近晶体表面存在着扩散层(结晶区)。物质在扩散层中的浓度和在整个溶液中的浓度是有差别的。这一差别的程度取决于分配系数$k_0$的大小以及晶体生长的速度(对于给定物质,主要取决于后者)。在使用式(3.28)时,$c_l(e)$应该是物质在扩散层中的浓度,实际上这一浓度是无法测量的,因此引入有效分配系数$k_\text{eff}$的概念,定义如下
\begin{equation} k_\text{eff} = \frac{c_s}{c_l}, \end{equation}
$c_s$和$c_l$分别是物质在固相和溶液中(远离生长界面)的实际浓度。显然,$k_\text{eff}$和晶体生长速度有关,而后者又取决于过饱和度和扩散层的厚度。当溶液中的浓度接近等于晶体界面附近溶液浓度时,即$c_l \approx c_l(e)$,则$k_\text{eff} \approx k_e$(因为在大多数情况下,$c_s \approx c_s(e)$)。

当单晶在溶液中生长时,由于晶体的各向异性,$k_\text{eff}$也和结晶方向有关。例如Fe$^{3+}$在KDP晶体的柱面和锥面上的分配系数并不一样,这样就给确定$c_s$和分配系数带来很多的困难。

总之,有效分配系数和晶体生长速度溶液的搅拌情况(影响扩散层厚度)以及结晶方向都有关系,如果测得的分配系数和上述几方面都关系不大时,我们就可认为体系接近于平衡状态,即$k_\text{eff} = k_e$。

有效分配系数还和物质的浓度有关。例如在重水溶液中生长DKDP晶体时,氘在晶体中的浓度$x$(以摩尔分数表示,$x=D/(D+H)$),取决于氘在溶液中的浓度(以摩尔分数$y$表示),其有效分配系数($k_\text{eff}(D) = x/y$)和$y$有关,关系式为
\begin{equation}
k_\text{eff}(D) = 0.68\exp(0.382y),
\end{equation}
D和H在溶液和晶体中的分布是均匀的,在氘化程度较高的溶液中,也可将D$_2$O看成溶剂,而把H$_2$O看成第三个组分(杂质)来计算氢的有效分配系数$k_\text{eff}(H)$
\begin{equation}
k_\text{eff}(H)=(1-x)/(1-y).
\end{equation}