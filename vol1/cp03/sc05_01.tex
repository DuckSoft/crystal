\subsection{凝胶法晶体生长特点}
与通常广泛应用的溶液法、熔体法、熔盐法和气相法等相比,凝胶法晶体生长是一种易于被人们忽视的方法,所使用的育晶设备虽然较为简单,但所涉及的物理化学过程变化多样,此方法本身具有下述的一些独特的优点。

\begin{enumerate}[(1)]\itemsep -0.5ex
\item 凝胶质软、性惰、多孔、半透明,且为不流动的半固体,凝胶介质有效地控制了流体的对流、湍流及外界扰动的发生。凝胶保持了化学上的惰性,且无害于晶体生长。凝胶骨架对晶体生长起到了“三维坩埚”支撑作用,它自然地服从于晶体生长形状,防止了晶体与容器底部或器壁的碰撞,使晶体保持完全无损的状态。
\item 凝胶具有微孔结构,有过滤、隔离成核和抑制成核的作用,因此不仅可使晶体降低杂质污染,并且有益于均匀成河以及均匀掺质晶体生长的研究。
\item 由于凝胶法晶体生长过程是在近于室温环境下进行的,故所生长出的晶体含有较少的热缺陷,从而提高了晶体的完整性。
\item 实际上,由于凝胶可以降低化学试剂的扩散速度,因此可以控制溶质的扩散速率和成核速率,有利于进行晶体生长基本理论的研究。
\item 凝胶法所用的育晶装置简便,化学试剂用量较少,生长的晶体品种较多,适用性很广,便于为研制新型功能晶体材料提供测试样品。
\end{enumerate}

事物总是一分为二的,凝胶法虽然有上述优点,但无疑也存有不足之处,如凝胶虽有抑制成核的作用,能减少非均匀成核的概率,但还很难保证在凝胶中仅有少数几个晶核成长,因此在一般情况下,生长线度为厘米级以上的晶体,除针状晶体外,当前还存在着一定的困难。
