\subsection{平衡和结晶过程的驱动力}
晶体生长是一个不平衡的过程。晶体在平衡状态时既不溶解也不生长,但要研究晶体生长过程,必须掌握有关该过程中平衡状态的知识。

溶液中生长晶体最重要的参数(平衡特征)是溶解度。溶解度曲线给出溶液的饱和浓度,即与固相处于平衡状态的溶液浓度。

可以把晶体生长看成是多相化学反应。当固体物质A在溶剂中溶解并达到饱和时,可用下述化学平衡方程式描述
\begin{equation}
\mathrm{A_{\text{固}} \rightleftharpoons A_{\text{溶解}}}
\end{equation}
\begin{equation}
K=\frac{[a]_e}{[a]_e(s)}
\end{equation}
其中$[a]_e$为饱和溶液中的平衡活度;$[a]_e(s)$为固相中的平衡活度,$K$为平衡常数。通常选取标准状态,使固体物质的活度等于1,此时,浓度单位最好用摩尔分数,这样上述方程就可在多组分的体系中应用。

组分活度$a_i$与其摩尔分数$x_i$的关系为
\begin{equation}
a_i = \gamma_i x_i
\end{equation}
$\gamma_i$是某组分$i$的活度系数,若$\gamma_i=1$,则该组分服从Raoult定律。如其活度系数不等于1,但仍是常数,则该组分服从Henry定律\footnote{Raoult定律:{$p=p_0x$}。{$p$}为溶液蒸汽压,$p_0$为纯溶剂蒸汽压,$x$为物质摩尔分数;Henry定律:$p=kx$,$p$为物质在溶液上的蒸汽压,$x$为物质摩尔分数,$k$为常数。}。

在理想溶液中,溶剂和溶质都服从Raoult定律,溶液的焓和体积分别等于溶剂和溶质的焓和体积之和。这意味着在溶解时,混合热和体积变化都等于零,但溶液的熵并不等于其组分熵的总和,因为熵是随无序性增加而增加的。在非理想的稀溶液中,溶质服从Henry定律,而溶剂则服从Raoult定律。

在3.1.1节中已提到,通常把溶液中含量较大的、即摩尔分数接近1的组分看作溶剂。浓溶液(溶剂和溶质含量差不多)并不服从Raoult定律和Henry定律,活度系数和浓度有关,需根据实验数据来估计。生长晶体的溶液大多是浓溶液,因此要考虑溶质质点间的相互作用,并在此基础上进行一些计算。

溶解度与温度的依赖关系可用Van't Hoff方程表示
\begin{equation}
\frac{\mathrm{d}\ln k}{\mathrm{d}T} = \frac{-\Delta H}{RT^2}.
\end{equation}
对溶液,上式变为
\begin{equation}
\frac{\mathrm{d}\ln [a]_e}{\mathrm{d}T} = -\frac{\overline{\Delta H}}{RT^2}.
\end{equation}
$\overline{\Delta H}$为溶质A在溶剂B中的偏摩尔焓的变化量,即表示将1摩尔溶质转移到体积足够大的溶液中时(此时溶液浓度不明显偏离饱和浓度)焓的变化、即溶解热(见3.1.6节)。在服从Raoult定律的溶液中,可用$x_A$来代替组分A的活度。

在溶液里生长晶体的过程中,自由能的变化为
\begin{equation}
\Delta G=\Delta G^0+RT\ln Q,
\end{equation}
其中
$$\Delta G^0=-RT\ln k,$$
$$Q=[a]^{-1},$$
$\Delta G^0$是生长过程中自由能的变化,$k$是式(3.19)的倒数,$[a]$是组分A在过饱和溶液中的实际活度。代入式(2.23)可得
\begin{equation}
\Delta G=RT\ln\frac{[a]_e}{[a]}
\end{equation}
其中$[a]_e$是组分A在饱和溶液中的平衡活度。若活度系数$\gamma=1$或是在一定范围内,$\gamma$和浓度无关,则在式(3.22)和式(3.24)中活度可用摩尔分数或浓度代替。因此式(3.24)中的$[a]_e/[a]$可用$c^*/c$代替。$c$和$c^*$分别是溶液在一定温度下的实际浓度和饱和浓度。根据式(3.8),$c/c^*$又是衡量过饱和度大小的过饱和比$s$,所以
\begin{equation}
\Delta G=RT\ln c^*/c = -RT\ln s
\end{equation}
从式(3.25)中可以看出,在过饱和溶液中($s>1$),生长晶体的过程是自由能降低的过程。这一过程是自动进行的。过饱和是结晶过程的驱动力。$s$愈大,自由能降低也愈多。晶体生长的驱动力也愈大。
