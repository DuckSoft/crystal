\chapter{晶体生长热力学}
\authors{唐棣生\quad 李沈军}
晶体生长是一门古老的“艺术”,但最近几十年来,由于热力学、统计物理以及其他学科在晶体生长中的应用,对解决晶体生长问题发挥了很大的作用,使晶体生长获得了牢固的科学基础,逐步发展成为材料科学中的一个重要分支,对解决工业与科研所需的材料问题做出了重要的贡献。因此,要想了解和掌握晶体生长这门学科,首先就必须掌握热力学的基本知识。

晶体生长是一个动态过程,不可能在平衡状态下进行,而热力学所处理的问题一般都是属于平衡状态问题。这两者结合到一起似乎有些矛盾。不过,在研究任何过程的动力学问题之前,对其中所包含的平衡问题有所了解,则可以预测过程中所遇到的问题(例如偏离平衡状态的程度),以及说明或提出解决问题的线索,因而在考虑实际晶体生长状况时,必须确定问题的实质究竟是与达到的平衡状态有关,还是与各种过程进行的速率有关。如果晶体生长的速率或晶体的形态取决于某一过程进行的速率(例如,在表面上的成核速率),那么就必须用适当的速率理论来分析,这时热力学就没有什么价值了。但如果过程进行程度非常接近于平衡态(准平衡态,这在高温时常常是如此),那么热力学对于预测生长量,以及成分随温度、压力和实验中其他变数而改变的情况,就有很大的价值。

实际上,可以认为晶体生长是控制物质在一定的热力学条件下进行的相变过程。

\section{相平衡及相变}
所谓相,是指体系中均匀一致的部分,它与别的部分有明显的分界线。例如,在大气压力下,冰与水混合共存时,冰本身是均匀一致的,它与水总有一定的分界面。不管是一大块冰,还是许多小冰块,冰所表现的物理状态和和化学状态总是一样的。所以在这个体系中,冰是一个相,水是另一个相。有时在同一种物质的固体状态(例如冰)下,由于温度或/和压强的不同,也可能有不同的结构(对称性)存在。这时候同时一种物质在固态下的不同结构,我们认为也是属于不同的相。但是在气体状态时,无论是由单一种气体或是几种不同气体组成,都是单一个相。

两相处于平衡状态时,其宏观性质有什么关系呢?我们可以根据热力学中的平衡判据来求出。

\subsection{热平衡}
设将$A$和$B$两个相封闭在一个与环境隔绝(与环境无热量和物质的交换)的体系内,$A$与$B$两相之间只有热量交换,即$A,B$两相间的隔板完全固定,只能导热。如图1.1所示,设此时从$A$有微量的热传到$B$内,则$A,B$两相的内能变化为
\begin{equation}
\begin{aligned}
\mathrm{d}U_A-T_AdS_A-P_AdV_A, \\
\mathrm{d}U_B-T_BdS_B-P_AdV_B
\end{aligned}
\end{equation}
\section{相图}

\subsection{单元系相图}
 %单元系相图
\subsection{二元系相图}
 %二元系相图
\subsection{三元系相图}
 %三元系相图
\subsection{参阅或绘制相图时的注意事项}
 %参阅或绘制相图时的注意事项

\section{相图在晶体生长中的应用}

\subsection{生长配料}
 %生长配料
\subsection{生长方式的选择}
 %生长方式的选择
\subsection{晶体完整性与生长速率}
 %晶体完整性与生长速率
\subsection{生长后的热处理工艺}
 %生长后的热处理工艺
\subsection{相图在晶体生长中的一个应用实例}
 %相图在晶体生长中的一个应用实例

