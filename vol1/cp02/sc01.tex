\section{晶体生长形态}
人们自17世纪初就开始了对矿物晶体和天然结晶的形态研究,几个世纪以来,通过对晶体形态的反复观察、分析与比较,归纳出一系列有关晶体形态(几何)的经验定律,诸如晶面角守恒定律、有理指数定律、晶体对称性定律、晶带定律等,这些经验点那个了吧为后来的晶体学发展奠定了学科基础。晶体生长形态是其内部结构的外在反映,晶体的各个晶面间的相对生长速率决定了他的生长形态。晶体生长形态虽受其内部结构的对称性、结构基元间键合和晶体缺陷等因素的制约,但在很大程度上还受到生长环境相的影响。因此,同一品种的晶体(即成分与结构均相同),既能形成具有对称特征的几何多面体,又能生长成特殊的形态。晶体生长形态能部分地反应出它的形成历史,因此,研究晶体生长形态,有助于人们认识晶体生长动力学过程,可为探讨实际晶体生长机制提供线索。

广义来讲,对于晶体生长形态学,不仅要研究晶体生长后的宏观几何外形以及其生长过程中宏观几何外形的演变,而且还应包括生长界面的显微形态和生长界面的稳定性等内容,这些研究内容,我们在下面将进行分析与讨论。

\subsection{晶体生长形态与生长速率间的联系}
一般说来,晶体在自由的生长体系中生长,晶体的各晶面生长速率是不同的,即晶体的生长速率是各向异性的。通常所说的晶体的晶面生长速率$R$是指在单位时间内晶面$(hkl)$沿其法线方向向外平行推移的距离($d$),并称为线性生长速率。

晶体生长的驱动力来源于生长环境相(汽相、溶液、熔体)的过饱和度($\Delta c$),或过冷度($\Delta T$)。人工生长单晶时,在保证晶体生长质量的前提下,人们总是希望提高生长速率,但是要维持恒定的生长速率,其工艺技术是很难达到的。常常由于晶体生长速率的改变,导致晶体缺陷的产生,这不仅有损于晶体的完整性,而且晶体的生长形态也要发生变化的。

晶体生长形态的变化来源于个镜面相对生长速率(比值)的改变,现以二维模式晶体生长为例来说明晶面的相对生长速率的变化与晶体生长形态间的联系(如图2.1所示)。

(图2.1)

在图2.1中,$l_{11},l_{01}$分别代表(11),(01)晶面的大小,$R_{11},R_{01}$分别代表(11),(01)晶面的生长速率。从图2.1所表明的简单的几何关系中可求得
\begin{align*}
R_{01}&=\frac{l_{01}}{2}+\sqrt{2}\cdot\frac{l_{11}}{2},\\
\frac{l_{01}}{2}&=\sqrt{2}\cdot R_{11}-R_{01},\\
R_{11}&=\sqrt{2}\cdot R_{01}-\frac{l_{11}}{2},\\
\frac{l_{11}}{2}&=\sqrt{2}\cdot R_{01}-R_{11}
\end{align*}
\begin{equation}
\frac{l_{01}}{l_{11}}=\frac{\displaystyle\sqrt{2}\left(\frac{R_{11}}{R_{01}}\right)-1}{\displaystyle\sqrt{2}-\frac{R_{11}}{R_{01}}}.
\end{equation}
根据式(2.1),当$R_{11}/R_{01}\geq\sqrt{2}$时,二维模式晶体生长形态仅为\{01\}单形;当$R_{11}/R_{01}\leq\sqrt{2}/2$时,二维模式晶体生长形态仅为\{11\}单形;当$\sqrt{2}/2<R_{11}/R_{01}<\sqrt{2}$时,二维模式晶体生长形态为\{01\}与\{11\}两种单形所组成的聚形。

同理,对于\{001\}与\{111\}两种单形所组成的立方晶系晶体形态,不难证明,它取决于(001)与(111)两晶面的相对生长速率的比值。 %晶体生长形态与生长速率间的联系
\subsection{晶体生长的理想形态}
具有几何形态的实际晶体,经过晶面角测量和极射赤平投影后,能够去伪存真,而描绘出晶体的理想形态。晶体的理想形态可分为单形和聚形。

当晶体在自由体系中生长时,若生长出的晶体形态的各个晶面的面网结构相同,而且各个晶面都是同形等大,这样的晶体理想形态称为单形。若在晶体的理想形态中,具有两套以上不同形、也不等大的晶面,这种晶体的理想形态称为聚形,而聚形是由数种单形构成的。

研究实际晶体生长形态,首先应当研究它的理想形态,以寻求晶面或晶带在三维空间分布的几何规律性,然后再进一步研究晶体生长形态出现的外在原因。 %晶体生长的理想形态
\subsection{晶体生长的实际形态}
 %晶体生长的实际形态
\subsection{晶体几何形态与其内部结构间的联系}
 %晶体几何形态与其内部结构间的联系
\subsection{环境相对晶体形态影响}
同一品种的晶体,由于晶体生长环境相的不同,往往会出现不同的晶体形态。现以自由生长体系为例来说明生长条件不同对晶体生长形态的影响。

\paragraph{(1)溶剂的影响}在过去漫长的年代里,人们对晶体形态的研究,大多花在矿物晶体和人工无机化合物晶体上,近20多年来,由于对有机非线性光学晶体的研究受到了重视,从而也开始注意到对有机晶体的生长形态的研究。现仅以{\CJKsetecglue{} 3-甲基-4-硝基吡啶-1-氧}晶体,(简称POM晶体),为例来说明溶剂对晶体形态的影响。

溶液法是研制块状有机非线性光学晶体的主要方法之一。与无机晶体水溶液生长不同之点是,有机晶体可选择多种不同有机非水溶剂。溶液中溶质与溶剂之间的相互作用对晶体生长过程有着极为重要的影响,因此,可从分子水平的晶体微观结构来研究{\CJKsetecglue{}溶质-溶剂}间的相互作用,研究晶体生长的基元化过程与晶体生长的脱溶剂化过程,进而研究晶体的生长机制与生长动力学规律,这些研究对有机晶体生长理论的发展和实际应用均有重要意义。

POM晶体时硝基吡啶类有机分子晶体,它比其他芳香硝基化合物的紫外截止波长更短,并具有较大的倍频系数,是研制非线性光学器件的候选材料。POM可溶于水、乙醇、甲苯、苯、丙酮、乙腈、环己酮、二甲基甲酰胺、二甲亚砜等溶剂中,因此,要想生长出优质大尺寸的POM晶体,优选最佳化的溶剂是一个很重要的生长条件。

溶质与溶剂间相互作用不仅影响溶液的溶解度,而且对晶体的生长形态会产生很大的影响。从环己酮、二甲基甲酰胺和二甲亚砜三种不同溶剂中生长出来的POM,其晶体形态如图2.6所示。

(图2.6)

从图2.6中可看出,POM晶体的生长形态主要是由$\{100\},\{210\},\{302\}$等单形所组成。之所以能出现不同的生长形态,可能是由于溶剂分子与某一晶面上溶质分子具有较强的选择吸附作用,难于脱溶剂化,从而降低了该晶面的生长速率,其结果便引起了晶体生长形态的变化。

\paragraph{(2)溶液pH值的影响}晶体从水溶液中生长的一个显著特点就是溶液pH值的变化对晶体生长形态有影响,控制溶液pH值的大小也是生长优质完整单晶的一个重要条件,现以$\rm\alpha-LiIO_3$晶体生长为例来加以说明。$\rm\alpha-LiIO_3$晶体是一种较典型的极性晶体,这种晶体在极轴两端的生长速率有明显的差异,在溶液的pH值为2.5、蒸发温度为70℃的生长条件下,快生长端生长速率$v_{[0001]}$是慢生长端生长速率$v_{[000\bar{1}]}$的2倍。但有的结果是$v_{[000\bar{1}]}$方向的生长速率是$v_{[0001]}$方向的3倍。因此,发生生长速率在正、负极方向反转的情况与溶液的pH值大小是有关的,在强酸溶液条件下,$\rm\alpha-LiIO_3$晶体的生长习性与中性溶液条件时是有明显的区别,在中性溶液中生长的$\rm\alpha-LiIO_3$晶体形态如图2.7所示。

但当溶液的pH值从高于pH临界值($\rm pH_c$)改变到低于$\rm pH_c$时,晶体生长的快慢端面发生倒转。

在$\rm pH<pH_c$的水溶液中生长$\rm\alpha-LiIO_3$的晶体形态,如图2.8所示。

(图2.7)(图2.8)

此时晶体的生长速率:$v_{[0001]}<v_{[000\bar{1}]}$,晶体形态是由六角锥面$\{10\bar{1}1\}$和$\{21\bar{3}2\}$以及六方柱面$\{10\bar{1}0\}$单形所组合的聚形。在$[0001]$端面方向有12个面,6个$\{10\bar{1}1\}$面呈五边形;6个$\{21\bar{3}2\}$呈四边形。在$[000\bar{1}]$端面方向,则仍由6个$\{10\bar{1}\bar{1}\}$面围成,每个面均呈三角形。

关于$\rm\alpha-LiIO_3$晶体的生长机制,有人曾提出过一些理论。诸如用离子电荷平衡以及$\rm Li^+$和$\rm IO_3^-$热骚动的差别来解释$\rm\alpha-LiIO_3$晶体生长速率的各向异性;从界面分子组浓度看$\rm\alpha-LiIO_3$晶体的机型生长;用Zata电势变号以及$\rm I_2$分子吸附的观点来解释$\rm\alpha-LiIO_3$晶体生长过程等理论。
 %环境相对晶体形态影响
