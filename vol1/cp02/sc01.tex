\section{晶体生长形态}
人们自17世纪初就开始了对矿物晶体和天然结晶的形态研究,几个世纪以来,通过对晶体形态的反复观察、分析与比较,归纳出一系列有关晶体形态(几何)的经验定律,诸如晶面角守恒定律、有理指数定律、晶体对称性定律、晶带定律等,这些经验点那个了吧为后来的晶体学发展奠定了学科基础。晶体生长形态是其内部结构的外在反映,晶体的各个晶面间的相对生长速率决定了他的生长形态。晶体生长形态虽受其内部结构的对称性、结构基元间键合和晶体缺陷等因素的制约,但在很大程度上还受到生长环境相的影响。因此,同一品种的晶体(即成分与结构均相同),既能形成具有对称特征的几何多面体,又能生长成特殊的形态。晶体生长形态能部分地反应出它的形成历史,因此,研究晶体生长形态,有助于人们认识晶体生长动力学过程,可为探讨实际晶体生长机制提供线索。

广义来讲,对于晶体生长形态学,不仅要研究晶体生长后的宏观几何外形以及其生长过程中宏观几何外形的演变,而且还应包括生长界面的显微形态和生长界面的稳定性等内容,这些研究内容,我们在下面将进行分析与讨论。

\subsection{晶体生长形态与生长速率间的联系}
一般说来,晶体在自由的生长体系中生长,晶体的各晶面生长速率是不同的,即晶体的生长速率是各向异性的。通常所说的晶体的晶面生长速率$R$是指在单位时间内晶面$(hkl)$沿其法线方向向外平行推移的距离($d$),并称为线性生长速率。

晶体生长的驱动力来源于生长环境相(汽相、溶液、熔体)的过饱和度($\Delta c$),或过冷度($\Delta T$)。人工生长单晶时,在保证晶体生长质量的前提下,人们总是希望提高生长速率,但是要维持恒定的生长速率,其工艺技术是很难达到的。常常由于晶体生长速率的改变,导致晶体缺陷的产生,这不仅有损于晶体的完整性,而且晶体的生长形态也要发生变化的。

晶体生长形态的变化来源于个镜面相对生长速率(比值)的改变,现以二维模式晶体生长为例来说明晶面的相对生长速率的变化与晶体生长形态间的联系(如图2.1所示)。

(图2.1)

在图2.1中,$l_{11},l_{01}$分别代表(11),(01)晶面的大小,$R_{11},R_{01}$分别代表(11),(01)晶面的生长速率。从图2.1所表明的简单的几何关系中可求得
\begin{align*}
R_{01}&=\frac{l_{01}}{2}+\sqrt{2}\cdot\frac{l_{11}}{2},\\
\frac{l_{01}}{2}&=\sqrt{2}\cdot R_{11}-R_{01},\\
R_{11}&=\sqrt{2}\cdot R_{01}-\frac{l_{11}}{2},\\
\frac{l_{11}}{2}&=\sqrt{2}\cdot R_{01}-R_{11}
\end{align*}
\begin{equation}
\frac{l_{01}}{l_{11}}=\frac{\displaystyle\sqrt{2}\left(\frac{R_{11}}{R_{01}}\right)-1}{\displaystyle\sqrt{2}-\frac{R_{11}}{R_{01}}}.
\end{equation}
根据式(2.1),当$R_{11}/R_{01}\geq\sqrt{2}$时,二维模式晶体生长形态仅为\{01\}单形;当$R_{11}/R_{01}\leq\sqrt{2}/2$时,二维模式晶体生长形态仅为\{11\}单形;当$\sqrt{2}/2<R_{11}/R_{01}<\sqrt{2}$时,二维模式晶体生长形态为\{01\}与\{11\}两种单形所组成的聚形。

同理,对于\{001\}与\{111\}两种单形所组成的立方晶系晶体形态,不难证明,它取决于(001)与(111)两晶面的相对生长速率的比值。 %晶体生长形态与生长速率间的联系
\subsection{晶体生长的理想形态}
 %晶体生长的理想形态
\subsection{晶体生长的实际形态}
 %晶体生长的实际形态
\subsection{晶体几何形态与其内部结构间的联系}
\paragraph{(1)晶体几何形态的表示方式}根据晶体学有理指数定律,晶体几何形态所出现的晶面符号$(hkl)$或晶棱符号$[hvw]$是一组互质的简单整数。

按照Bravais法则,当晶体生长到最后阶段而保留下来的一些主要晶面是具有面网密度较高,而面网间距$d_{hkl}$较大的晶面。

晶体生长形态的变化,除同质多相的晶体外,不仅与晶体生长条件有关,而且也能反映出晶体结构的一些信息。

从X射线晶体结构分析结果可知,不管是高级晶系或是中、低级晶系晶体,晶格面网间距$d_{hkl}$、晶格常数($a,b,c,\alpha,\beta,\gamma$)和面网族$\{hkl\}$三者之间存在着一定的关系,例如,对于立方面心晶格晶体,面网间距$d_{hkl}$、晶格常数$a$和面网族$\{hkl\}$三者之间存在着如下关系:

当$h,k,l$全为奇数或全为偶数时
\begin{equation}
d_{hkl}=\frac{a}{\sqrt{h^2+k^2+l^2}}
\end{equation}

当$h,k,l$中有奇数也有偶数时
\begin{equation}
d_{hkl}=\frac{a}{2\sqrt{h^2+k^2+l^2}}
\end{equation}
从式(2.2)和式(2.3)中可得出,$a^2/d^2_{hkl}$值随$h,k,l$值变化是有规律性的,见表2.1。

(表2.1\quad$a^2/d^2_{hkl}$值随$h,k,l$值的变化规律)

根据上述Bravais法则,属于这种结构的晶体,出现在晶体形态中的单形顺序应为$\{111\},\{100\},\{110\},\cdots$。天然的萤石($\rm CaF_2$)和金刚石(C)等晶体,基本上是符合上述规律的。但对于中、低级晶系晶体,尤其是对于低级晶系晶体,面网间距$d_{hkl}$、晶格常数($a,b,c,\alpha,\beta,\gamma$)和面网族$\{hkl\}$三者之间的关系就复杂化了,但$d_{hkl}$值随着面网族$\{hkl\}$的减小而增大的一般倾向却仍然存在。更值得注意的是,当晶体结构中存在着螺旋轴和滑移面对称性时,则情况变得更加复杂了。这时候必须对面网间距$d_{hkl}$进行修正,否则计算结果与实际情况就会表示不符。

当晶体结构中具有螺旋轴时,只影响与其垂直的面网间距$d_{hkl}$值,而修正因子$\beta$应随螺旋轴的轴次不同而不同,对于:
\begin{align*}
&2_1,4_2,6_3\text{ 螺旋轴}, &\beta = 1/2, \\
&3_1,3_2,6_2,6_4\text{ 螺旋轴}, &\beta = 1/3,\\
&4_1,4_3\text{ 螺旋轴}, &\beta = 1/4,\\
&6_1,6_5\text{ 螺旋轴}, &\beta = 1/6.
\end{align*}

例如属于三方晶系的石英($\alpha\text{-}\rm SiO_2$)晶体,若不考虑其结构中的$3_1$或$3_2$螺旋轴时,根据Bravais法则的推论,$z\{0001\}$单形应在晶体形态中出现的比重最大,但实际上所出现的概率却很少,若考虑到垂直于$z\{0001\}$单形的$3_1$或$3_2$螺旋轴的重要作用时,而$z\{0001\}$单形应改为$z\{0003\}$单形,这样对于$z\{0001\}$单形很少出现的原因,就可以得到合理的解释。

同样,当晶体结构中存在着滑移面时,对晶体形态中所出现的单形比重次序也能发生影响,可以证明,与(001)面平行的滑移面,只能对$\{hkl\}$单形范围内的某些单形发生影响。例如$a$滑移面或$b$滑移面,仅对$h$或$k$为奇数的$\{hko\}$单形产生影响,这由于等同部分形成的面网间距$d_{hkl}$与密度都比具有普通对称面存在时少$1/2$的原因\footnote{此处费解。}。同样,不难理解,当晶体结构中具有$n$滑移面时,仅对$(h+k)$为奇数的$\{hko\}$单形产生影响,而对$d$滑移面,仅对$(h+k)$不等于4的倍数的$\{hko\}$单形产生影响。

上述所讨论的晶体形态与晶体结构间的联系只能看作是一个粗略的轮廓,它是晶体结构对称性在其形态上反映的一些信息,但在实际情况下,由于晶体生长的外界因素及其结构基元间键合作用的影响,即便是同一品种的晶体生长形态往往也会有所不同,这样就远非像Bravais法则所推论的那样简单了。

\paragraph{(2)周期键链理论(periodic bond chain, PBC)} Hartman和Perdok等在探索晶体形态与其结构的关系时,提出了周期键链理论,此理论是在晶体化学基础上建立起来的晶体形态理论,对于具有复杂结构的晶体实验观察表明,其生长形态是可以用周期键链理论来阐明的。此理论的基本假设是,在晶体生长过程中,于生长界面上形成一个键所需要的时间随着键合能的增加而减少,因而生长界面的法向生长速率随键合能的增加而增加。由于键合能的大小决定了生长界面的法向生长速率,故键合能的大小也就决定的晶体的生长形态。该理论认为晶体结构是由周期键链(PBC)所组成的,晶体生长最快的方向是化学键最强的方向,晶体生长是在没有中断的强键链存在的方向上,这里所说的强键是在晶体生长过程中形成的强键。晶体生长过程所能出现的晶面可划分为三种类型,即$F$面、$S$面、$K$面。划分面的标准为

$F$面:或称平坦面(flat faces),它包含两个或两个以上的共面的PBC($\bm{PBC}$矢量)。

$S$面:或称台阶面(stepped faces),它包含一个PBC($\bm{PBC}$矢量)。

$K$面:或称扭折面(kinked faces),它不包含PBC($\bm{PBC}$矢量)。

所设想的PBC模型,如图2.2所示。

在图2.2中,假设晶体中具有三种$\bm{PBC}$矢量,其中$\bm{A}\text{ 矢量}\parallelsum [100]$、$\bm{B}\text{ 矢量}\parallelsum [010]$、$\bm{C}\text{ 矢量}\parallelsum [001]$方向。这些$\bm{PBC}$矢量确定了六个$F$面,即$(001),(00\bar{1});(010),(0\bar{1}0);(100),(\bar{1}00)$面;三个$S$面即$(011),(101)$和$(110)$面;一个$K$面,即$(111)$面。从图2.2中还可看出,一个结构基元生长在$F$面上,只形成一个不在$F$面上的$\bm{PBC}$矢量;一个结构基元生长在$S$面上,形成的强键比$F$面上的数目多;而在$K$面上形成的强键数最多。因此,$F$面的生长速率最慢,$S$面的生长速率次之,而$K$面生长速率最快,因而$K$面是易于消失的晶面。晶体生长的最终形态多为$F$面包围,其余的为$S$面。

 %晶体几何形态与其内部结构间的联系
\subsection{环境相对晶体形态影响}
同一品种的晶体,由于晶体生长环境相的不同,往往会出现不同的晶体形态。现以自由生长体系为例来说明生长条件不同对晶体生长形态的影响。

\paragraph{(1)溶剂的影响}在过去漫长的年代里,人们对晶体形态的研究,大多花在矿物晶体和人工无机化合物晶体上,近20多年来,由于对有机非线性光学晶体的研究受到了重视,从而也开始注意到对有机晶体的生长形态的研究。现仅以{\CJKsetecglue{} 3-甲基-4-硝基吡啶-1-氧}晶体,(简称POM晶体),为例来说明溶剂对晶体形态的影响。

溶液法是研制块状有机非线性光学晶体的主要方法之一。与无机晶体水溶液生长不同之点是,有机晶体可选择多种不同有机非水溶剂。溶液中溶质与溶剂之间的相互作用对晶体生长过程有着极为重要的影响,因此,可从分子水平的晶体微观结构来研究{\CJKsetecglue{}溶质-溶剂}间的相互作用,研究晶体生长的基元化过程与晶体生长的脱溶剂化过程,进而研究晶体的生长机制与生长动力学规律,这些研究对有机晶体生长理论的发展和实际应用均有重要意义。

POM晶体时硝基吡啶类有机分子晶体,它比其他芳香硝基化合物的紫外截止波长更短,并具有较大的倍频系数,是研制非线性光学器件的候选材料。POM可溶于水、乙醇、甲苯、苯、丙酮、乙腈、环己酮、二甲基甲酰胺、二甲亚砜等溶剂中,因此,要想生长出优质大尺寸的POM晶体,优选最佳化的溶剂是一个很重要的生长条件。

溶质与溶剂间相互作用不仅影响溶液的溶解度,而且对晶体的生长形态会产生很大的影响。从环己酮、二甲基甲酰胺和二甲亚砜三种不同溶剂中生长出来的POM,其晶体形态如图2.6所示。

(图2.6)

从图2.6中可看出,POM晶体的生长形态主要是由$\{100\},\{210\},\{302\}$等单形所组成。之所以能出现不同的生长形态,可能是由于溶剂分子与某一晶面上溶质分子具有较强的选择吸附作用,难于脱溶剂化,从而降低了该晶面的生长速率,其结果便引起了晶体生长形态的变化。 %环境相对晶体形态影响
