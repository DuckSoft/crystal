\subsection{研究界面稳定性应遵循的几个原则}
晶体生长是一种相变过程。晶体从熔体中生长,熔体沿着运动的相界面转化为晶体。晶体从溶液中生长,通过溶质脱溶剂化,然后吸附在生长界面,再进入生长位置。这样在界面附近就必然要发生热量与质量输运,此关系到界面的稳定性。研究界面稳定性应遵循的几个原则为

\begin{enumerate}[(1)] \itemsep -0.5ex
\item 界面上能量(热)守恒
\begin{equation}
Lv\cdot\hat{n}=(-\kappa_l\nabla T_l+\kappa_s\nabla T_s)\cdot\hat{n},
\end{equation}
式中$L$为熔体单位体积的熔化潜热,$v$为晶体生长速率,$\hat{n}$为垂直于生长界面并指向熔体的单位长度,$\cdot$为标量积,$\kappa_l$为熔体的热导率,$\nabla=\dfrac{\partial}{\partial x}i+\dfrac{\partial}{\partial y}j+\dfrac{\partial}{\partial z}k$(哈密顿算符),$T_l$为熔体温度,$\kappa_s$为固相的热导率,$\T_s$为固相温度。

\item 界面上溶质守恒
\begin{equation}
(c_l-c_s)v\cdot\hat{n}=-D_l\nabla c_i\cdot\hat{n},
\end{equation}
式中$c_l$为溶质在液相中的浓度,$c_s$为溶质在固相中的浓度,$D_l$为溶质的扩散系数。

\item 界面温度的连续性
\begin{equation}
T_l=T_s,
\end{equation}
$T_l$与$T_s$分别为界面两侧的液相与固相的温度。

\item 界面上温度与组分间的热力学关系。相界面上的温度与组分所偏离的热力学平衡值,为界面动力学过程提供了驱动力。当偏离的平衡值很大时,则界面动力学过程起支配作用,例如稳定的小晶面生长就是其中一例。当偏离的平衡值可以忽略不计时,输运过程起支配作用。生长界面是否稳定,主要受两种重要因素的支配,一个为界面附近的温度梯度,另一个为溶质浓度梯度,并且这两中因素又存在着相互关联的关系。

\qquad 根据二元系相图可知,在相平衡的条件下,当溶质(作为一组元)的平衡分凝系数$k_0<1$时,在界面上的溶质被排除,从而促使溶液的凝固点降低,但当$k_0>1$时,溶质优先地进入晶体,促使溶液的凝固点升高。因此,溶液的凝固点是溶质浓度的函数。如果溶液中溶质的含量是微小的,则可将溶液的凝固点和溶质浓度间的关系金丝看成是线性关系,于是平坦界面溶液的凝固点与溶质浓度之间的关系可用下式表示:
\begin{equation}
T_i=T_0+mc_l
\end{equation}
式中$T_0$为纯溶剂的凝固点,$m$为液相线斜率,即$m=dT/dc_1$,此表示溶液中溶质改变单位浓度所引起的凝固点的变化,其数值取决于溶液的性质,对于提高溶液凝固点的溶质,即$k_0>1$时,$m$为正值;对于降低凝固点的溶质,即$k_0<1$时,$m$为负值。对于稀薄溶液体系,$m$可看作为常数。$c_l$为溶液中溶质的浓度。

\qquad 对于曲界面,由于存在着Gibbs-Thomson效应,溶液的凝固点与其溶质浓度间的关系可用下式表示:
\begin{equation}
T_i=T_0+mc_l-T_0\Gamma(\frac{1}{r_1}+\frac{1}{r_2}),
\end{equation}
式中右面的第一项为纯熔体(溶剂)的凝固点,第二项为溶质所引起的凝固点的变化,第三项为界面曲率所引起的凝固点的改变,$r_1,r_2$为界面上给定点的主曲率半径,$\Gamma=\sigma/L$,$\sigma$为界面的比表面自由能,$L$为单位体积的熔化潜热,通常人们称$T_0\Gamma$为Gibbs-Thomson系数。

\qquad 由式(2.37)可知,当界面为平坦面时,$k=\dfrac{1}{r_1}+\dfrac{1}{r_2}=0$时,该式即简化为式(2.36)。
\end{enumerate}