\subsection{输运的类型}
晶体生长的输运类型,主要包括热量、质量和动量等类型,现分别加以阐明。

\paragraph{(1)热量输运}从熔体中生长单晶,主要靠热量输运来实现。晶体生长靠体系中的温度梯度所造成的局部过冷来驱动,只要体系中存在着温度梯度,就会产生热量输运。温度梯度是一个矢量,它的方向是沿着温场的等温面法线从低温指向高温,而热量总是从高温向低温传递,即热量总是沿着温度梯度相反的方向输运。控制热量传输,提供一个合适而稳定的温场(温度的空间分布),使来自熔体的热量与结晶潜热从固液界面处连续不断的输运出去,从而保证单晶能够稳定而正常生长。显然,这是生长高质量单晶的一个关键因素。

晶体生长过程中的热量输运,主要通过三种方式来进行,即辐射、传导和对流。在晶体生长过程中,哪一种热传递起主要作用,必须根据具体的工艺条件来定。一般来说,在高温时,界面处传递出去的大部分热量是从晶体表面辐射出去,而传导和对流则起次要的作用。在低温时,热量输运主要靠传导来进行。对于生长高熔点氧化物单晶,例如石榴石型晶体,辐射和传导是热量输运的两种主要形式。

如果熔体的热量输运纯属于热传导的作用,这时与热量输运相对应的热传导方式可用下式表示:
\begin{equation}
\frac{\partial T}{\partial t}=K\Delta T=K(\frac{\partial^2T}{\partial x^2}+\frac{\partial^2T}{\partial y^2}+\frac{\partial^2T}{\partial z^2}).
\end{equation}
式中,$K$为热传导系数,$\Delta T$为熔体中的温度差值,$t$为时间。

如果将熔体的物理常数(如密度、热容、热传导系数等)随温度而变化的值忽略不计,也不考虑对流传热所引起的能量消耗,那么熔体的对流传热方程可写为下式:
\begin{equation}
\frac{\partial T}{\partial t}+\rho c_pv\nabla T=K\Delta T,
\end{equation}
$$\nabla T=\frac{\partial T}{\partial x}i+\frac{\partial T}{\partial y}j+\frac{\partial T}{\partial z}k$$
为温度梯度,式中,$v$为熔体的流动速度,$\rho$为熔体的密度,$c_p$为熔体的定压热容。
在恒稳条件下,即$\dfrac{\partial T}{\partial t}=0$,那么式(2.5)便可进一步简化为
\begin{equation}
\rho c_pv\nabla T=K\Delta T.
\end{equation}
如果熔体处于静止状态,即$v=0$,那么,式(2.5)就变成上述热传导方程式,即变成式(2.4)。

\paragraph{(2)质量输运}晶体生长的质量输运存有两种截然不同的输运模式,其一为扩散,其二为对流,前者是通过分子运动来实现的,后者是通过溶解于流体中的物质质点,在流体宏观运动过程中被流体带动并一同输运。在实际晶体生长输运过程中,对流与扩散起着同等重要的作用。

在溶液中溶质的浓度随空间位置而变化,每一点都有确定的浓度,而不同点的浓度不完全相同,通常把溶质在溶液中浓度的空间分布称为溶质的浓度场。在浓度场中,将溶质浓度祥等的空间各点联结起来,所得到的空间曲面称为等浓度面。浓度梯度是一种矢量,它沿着等浓度面的法线并指向浓度升高的方向,其大小是沿该方向单位长度浓度的变化。扩散的驱动力来源于溶液浓度梯度。若浓度场记为$(x,y,z)$,则浓度梯度可表示为
\begin{equation}
\nabla c_i=\frac{\partial c_i}{\partial x} \bm{i}+ \frac{\partial c_i}{\partial y} \bm{j}+ \frac{\partial c_i}{\partial z} \bm{k}.
\end{equation}
溶质扩散起因于浓度梯度。恒稳态下,描述扩散过程的数学基础是Fick的第一、第二扩散定律。当第一扩散定律用于一维扩散时,可用下式表示:
\begin{equation}
J_x=-D_{Li}\frac{\partial c_i}{\partial x}
\end{equation}
式中的$J_x$为物质体系中第$i$个组分的通量(单位时间内通过单位面积的物质量),又称物质流密度,$D_{Li}$为第$i$个组分在溶液中的扩散系数,$\dfrac{\partial c_i}{\partial x}$为第$i$个组分在$x$方向上的浓度梯度,右面的负号表示溶质扩散沿着溶质浓度降低的方向进行。式$(2.8)$只能适用于恒稳态的扩散,$\dfrac{\partial c_i}{\partial x}$不随试键的变化而变化,但在实际晶体生长过程中,一般不具备恒稳态的条件,而$c_i$与$\dfrac{\partial c_i}{\partial x}$均为距离与时间的函数。在这种情况下,可应利用Fick的第二扩散定律。当第二扩散定律用于一维扩散时,可表示为
\begin{equation}
\frac{\partial c_i}{\partial t} = D_{Li}\frac{\partial^2c_i}{\partial x^2}.
\end{equation}
当用于三维扩散时,可用下式表示:
\begin{equation}
\frac{\partial c_i}{\partial t} = D_{Li}(\frac{\partial^2c_i}{\partial x^2}+\frac{\partial^2c_i}{\partial y^2}+\frac{\partial^2c_i}{\partial z^2}).
\end{equation}

扩散系数$D_{Li}$一般都作为常数,但严格地来讲,它并不是常熟,而是溶质浓度的函数,而且$D_{Li}$的数值取决于实验条件,在一定的温度下,$D_{Li}$是从一系列溶质浓度中所测量出来的平均值,并称它为积分扩散系数。如果只是从一个溶质浓度中测量出的数值,则称为微分扩散系数。文献中所记载的许多物质的$D_{Li}$值既不是积分扩散系数,也不是微分扩散系数,而是根据$D_{Li}$与溶质浓度无关的假设而计算出来的数值,它是表征气相或溶液输运性质的重要参数,$D_{Li}$的单位为$\mathrm{cm^2/s}$。

关于对流输运的问题,由于生长晶体流体(气\footnote{原文作“汽”。}体、溶液)本身的组成往往是多组元的,因此描述晶体生长的对流输运,应该是混合流体的动力学方程式。为了使问题简化起见,只考虑不可压缩的定常流体,这样,物质的对流扩散方程可表示为
\begin{equation}
\frac{\partial c_i}{\partial t}+\bm{v}\nabla c_i=D_{Li}\Delta c_i.
\end{equation}
如式$(2.11)$用分量表示,可写成下式:
\begin{equation}
\frac{\partial c_i}{\partial t}+v_x\frac{\partial c_i}{\partial x}+v_y\frac{\partial c_i}{\partial y}+v_z\frac{\partial c_i}{\partial z}=D_{Li}(\frac{\partial^2c_i}{\partial x^2}+\frac{\partial^2c_i}{\partial y^2}+\frac{\partial^2c_i}{\partial z^2}).
\end{equation}
这里$\bm{v}$为流体的流动速度,一般来说,$\bm{v}$是坐标的函数。

如果溶质浓度$c_i$在流体中的分布,不随时间$t$而改变,即流体处在恒稳态下,这时$\frac{\partial c_i}{\partial t}=0$,那么,式$(2.12)$便变成
\begin{equation}
\bm{v}\nabla c_i=D_{Li}\Delta c_i
\end{equation}
如果流体处于静止状态,即$\bm{v}=\bm{0}$,因而式$(2.12)$就变成Fick第二扩散定律方程式,即变成式$(2.10)$。

%TODO: (3)动量输运
