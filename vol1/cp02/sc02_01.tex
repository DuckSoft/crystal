\subsection{输运的类型}
晶体生长的输运类型,主要包括热量、质量和动量等类型,现分别加以阐明。

\paragraph{(1)热量输运}从熔体中生长单晶,主要靠热量输运来实现。晶体生长靠体系中的温度梯度所造成的局部过冷来驱动,只要体系中存在着温度梯度,就会产生热量输运。温度梯度是一个矢量,它的方向是沿着温场的等温面法线从低温指向高温,而热量总是从高温向低温传递,即热量总是沿着温度梯度相反的方向输运。控制热量传输,提供一个合适而稳定的温场(温度的空间分布),使来自熔体的热量与结晶潜热从固液界面处连续不断的输运出去,从而保证单晶能够稳定而正常生长。显然,这是生长高质量单晶的一个关键因素。

晶体生长过程中的热量输运,主要通过三种方式来进行,即辐射、传导和对流。在晶体生长过程中,哪一种热传递起主要作用,必须根据具体的工艺条件来定。一般来说,在高温时,界面处传递出去的大部分热量是从晶体表面辐射出去,而传导和对流则起次要的作用。在低温时,热量输运主要靠传导来进行。对于生长高熔点氧化物单晶,例如石榴石型晶体,辐射和传导是热量输运的两种主要形式。

如果熔体的热量输运纯属于热传导的作用,这时与热量输运相对应的热传导方式可用下式表示:
\begin{equation}
\frac{\partial T}{\partial t}=K\Delta T=K(\frac{\partial^2T}{\partial x^2}+\frac{\partial^2T}{\partial y^2}+\frac{\partial^2T}{\partial z^2}).
\end{equation}
式中,$K$为热传导系数,$\Delta T$为熔体中的温度差值,$t$为时间。

如果将熔体的物理常数(如密度、热容、热传导系数等)随温度而变化的值忽略不计,也不考虑对流传热所引起的能量消耗,那么熔体的对流传热方程可写为下式:
\begin{equation}
\frac{\partial T}{\partial t}+\rho c_pv\nabla T=K\Delta T,
\end{equation}
$$\nabla T=\frac{\partial T}{\partial x}i+\frac{\partial T}{\partial y}j+\frac{\partial T}{\partial z}k$$
为温度梯度,式中,$v$为熔体的流动速度,$\rho$为熔体的密度,$c_p$为熔体的定压热容。
在恒稳条件下,即$\dfrac{\partial T}{\partial t}=0$,那么式(2.5)便可进一步简化为
\begin{equation}
\rho c_pv\nabla T=K\Delta T.
\end{equation}
如果熔体处于静止状态,即$v=0$,那么,式(2.5)就变成上述热传导方程式,即变成式(2.4)。