\section{晶体生长界面的稳定性}
晶体生长界面的稳定性涉及到晶体质量的优劣,当生长优质块状大单晶时,相变必须在稳定的生长界面上发生,以便保持晶体结构基元排列的均一性。

多少年来,晶体生长界面稳定性理论一直是晶体生长领域中的一个重要研究课题,该理论实际上是将研究流体动力学稳定性的一套数学方法,可用来解决晶体生长过程中的界面稳定性问题\footnote{费解。}。

晶体生长过程中的界面是否稳定,直接关系到晶体生长的形态。从自由生长体系中生长晶体(诸如从气相或溶液中生长晶体),在过去较长的时期内,人们大多从晶体几何外形来研究晶体形态与其外界生长条件间的关系。从强制生长体系中生长晶体(诸如熔体提拉、坩埚下降法生长晶体),人们最为关心的问题是固液界面的形状(凸形、凹形还是平坦)、在界面上有无小晶面出现等,这些问题涉及到界面稳定性理论。这里所说的界面稳定性是以宏观尺寸来度量的,界面起伏一般在10---100$\mu$m,因此不直接涉及原子尺度界面的光滑、粗糙和成核动力学理论,如果说涉及到原子尺度,通常仅为晶体生长过程中的化学元素成分。由界面不稳定性所显示的生长特种,对于透明晶体,可在光学显微镜下直接观察界面的形态变化;对于不透明晶体,经常可借助于晶体结构与组分变化的测定来鉴定。

在晶体生长过程中,当生长界面从宏观尺度来看是光滑的,但偶尔受到温度、溶质浓度的起伏、或外因干扰时,如果这些干扰随着时间的推移和晶体生长过程的延续而逐渐衰减,最后自动消失,同时界面也随着恢复到原来光滑水平面的状态,这种界面则是稳定的。如果随着晶体生长过程的延续,这些偶尔产生的干扰逐渐增大,使界面产生凸凹不平的状态,这种界面则是不稳定的。

\subsection{研究界面稳定性应遵循的几个原则}
 %研究界面稳定性应遵循的几个原则
\subsection{生长界面稳定性的判据}
生长晶体时为了使界面稳定,寻求界面稳定性的判据显然是很重要的,确定界面是否稳定,可通过界面附近熔体的温度梯度、溶液中溶\footnote{原文作“熔”。}质的浓度梯度和界面效应等途径来做出判断的。

\paragraph{(1)熔体的温度梯度}晶体生长界面前沿的温度梯度,不外乎三种情况,一种是正温度梯度,即$(dT_l/dx)>0$,$x$的方向指向熔体,这样的熔体称为过热熔体。另一种情况称为负温度梯度,即$(dT_l/dx)<0$,这样的熔体称为过冷熔体。第三种情况是不常见的,即界面前沿的温度为熔体熔点温度,这时$\dfrac{dT_l}{dx}=0$。

对于过热熔体,如果平坦界面在偶然的外界因素干扰下而出现了凹凸不平,但由于离开界面的熔体温度梯度为正值,界面的凸起部位必然处于较高的温度$T_l$。由于$T_l>T_0$($T_0$为熔体的凝固点温度),在这种情况下,界面凸起部分的生长速率便会逐渐降低,从而被界面凹入部位的生长所追及,最后生长界面必然恢复到原来的光滑面的状态,因此这种类型的界面是稳定的,这样,熔体中的正温度梯度是有利于界面稳定性的因素,并可作为生长界面稳定性的判据。根据界面上能量(热)守恒原则
$$\kappa_s\frac{\partial T_s}{\partial x}=\kappa_l\frac{\partial T_l}{\partial x}+f\rho L$$
由于$\dfrac{\partial T_l}{\partial x}>0$,可用生长速率$f$的大小来作为界面稳定性的判据
\begin{equation}
f<\frac{\kappa_s}{L\rho}\frac{\partial T_s}{\partial x},
\end{equation}
式中$\rho$为晶体密度,$L$为结晶潜热。

对于过冷的熔体,熔体中的温度梯度为负值,即$(\partial T_l/\partial x)<0$,这时熔体中的温度$T_l$低于熔点温度$T_0$,在这种情况下,有利于平坦界面受外界因素干扰而产生的凸起部位的生长,这种类型的界面显然是不稳定的,因此熔体中的负温度梯度是不利于界面稳定性的因素。

当熔体中的温度等于熔点温度时,即$(\partial T_l/\partial x)=0$,整个熔体温度时均匀分布,在这种情况下,平坦界面是否稳定,那就要看平坦界面所受外界干扰大小而定,当干扰大时,平坦界面也能变为不稳定的。

 %生长界面稳定性的判据
\subsection{界面稳定性动力学理论}
早在1964年,Mullins和Sekerka运用干扰技术检验了球形界面和平坦界面的稳定性,从而奠定了界面稳定性理论的基础。几十年来,界面稳定性动力学理论一直是一个重要的研究领域。

检验界面稳定性的干扰技术,其基本思想如下:由扩散方程求得的运动界面的运动方程,叠加一个微小的形态干扰函数,这样就得到了受干扰后的运动界面方程。人们可根据干扰后的界面满足于扩散方程这一点来确定形态干扰函数。如果形态干扰函数的振幅确实随时间而变化,那么原来的界面就是不稳定的,否则就是稳定的,这就是检验界面稳定性的所谓干扰技术。

晶体生长过程中,移动的界面上不可避免的会出现干扰,界面上一旦出现几何干扰,也必然在界面附近引起局部温度场和浓度场的变化,这些可相应地理解为温度干扰和浓度干扰。干扰技术的实质在于研究热量或质量在扩散场中的干扰行为,即研究干扰振幅与时间的依赖关系。界面上出现的任何周期性干扰都可用正弦函数的傅里叶级数来表示,因此,我们应该考虑到所有可能波长的正弦干扰行为。例如在强制生长体系中,当界面未受到干扰时,它恒为等速运动的平面,在运动坐标系中,其界面方程为$Z\equiv 0$。当受到正弦式的几何干扰后,界面干扰形状可用下式表示:
\begin{equation}
Z(x,t)=\phi(x,t)=\delta(t)\sin(\omega x),
\end{equation}
式中$\delta(t)$为微干扰的振幅,$\omega$为微干扰的空间频率,$\lambda=\frac{2\pi}{\omega}$为微干扰的波长。通常,经过较长的时间后,微干扰振幅$\delta(t)$与时间的关系可表示为指数关系,即
\begin{equation}
\delta(t)\sim\text{常数}\cdot\exp(pt),
\end{equation}
于是有
\begin{equation}
p=\lim\frac{d\delta(t)}{dt}\cdot\frac{1}{\delta(t)}=\frac{\dot{\delta}}{\delta}
\end{equation}
式中$\dot{\delta}/\delta$称为单位振幅的变化率。

由式(2.44)可知,在界面上一旦出现微干扰,而干扰取决于$p$值,当$p$为正值时,$\delta(t)$便随时间而增大,界面是不稳定的;但当$p$为负值时,$\delta(t)$随时间而减弱,界面是稳定的。从式(2.45)中可以看出,$p$值的大小决定了$\delta(t)$的增长或衰减的速率。作为界面稳定性的判据是微干扰随时间而变化。如果$\dfrac{d\delta(t)}{dt}<0$,则界面是稳定的,如果$\dfrac{d\delta(t)}{dt}>0$,则界面是不稳定的,其相应地图形如图2.16所示。

(图2.16 界面受微干扰后的变化情况)

为了研究界面的稳定性,关键在于研究扩散场中所有波长的干扰振幅与时间的关系。干扰本身的行为受到扩散场的支配,因而,干扰后的温度场和浓度场必然满足与时间相关的扩散方程式。在充分长的时间后,温度干扰和浓度干扰振幅与时间的关系,与几何干扰所得的结论一样,界面是否是稳定的,主要取决于单位振幅的变率($\dot{\delta}/\delta$)。

上述所谈的微干扰界面稳定性理论描述了一具有任意波长的无限小干扰的界面增长和衰减,这对了解界面是否稳定是十分方便的。在一组给定的生长条件下,利用上述理论便可得到干扰增长或衰减的指数速率。这组给定的生长条件包括未受干扰界面的温度梯度、浓度梯度、热扩散率、溶质扩散率、潜热、宏观生长速率、界面能等。因此,这一线性动力学理论使人们对晶体生长过程的认识大大地深刻多了。

在晶体生长过程中,微干扰出现以后,界面邢台的变化远比干扰出现时的形态变化更为复杂,例如胞状界面的出现、枝蔓晶体生长是不能用线性动力学来描述的,而有待于用非线性界面动力学予以解释。

保持界面的稳定性,对生长优质、完整的晶体至关重要。晶体生长的实验证明,通常导致界面不稳定的因素主要是以下几个方面:(1)过小的温度梯度;(2)太快的生长速率;(3)太多的溶质(对熔体生长而言)或太多的溶剂(对溶液生长而言)。欲促使界面稳定,其补救的方法如下:可利用较大的温度梯度、较慢的生长速率、较小的溶质浓度(对熔体生长而言)或较浓的溶液来生长晶体,并要综合各种影响因素加以处理。 %界面稳定性动力学理论

\clearpage