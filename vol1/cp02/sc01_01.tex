\subsection{晶体生长形态与生长速率间的联系}

% 等待“永远第二的学生”的录入
【此处等待插入“永远第二的学生”的部分录入】

(图2.1)

在图2.1中,$l_{11},l_{01}$分别代表(11),(01)晶面的大小,$R_{11},R_{01}$分别代表(11),(01)晶面的生长速率。从图2.1所表明的简单的几何关系中可求得
\begin{align*}
R_{01}&=\frac{l_{01}}{2}+\sqrt{2}\cdot\frac{l_{11}}{2},\\
\frac{l_{01}}{2}&=\sqrt{2}\cdot R_{11}-R_{01},\\
R_{11}&=\sqrt{2}\cdot R_{01}-\frac{l_{11}}{2},\\
\frac{l_{11}}{2}&=\sqrt{2}\cdot R_{01}-R_{11}
\end{align*}
\begin{equation}
\frac{l_{01}}{l_{11}}=\frac{\displaystyle\sqrt{2}\left(\frac{R_{11}}{R_{01}}\right)-1}{\displaystyle\sqrt{2}-\frac{R_{11}}{R_{01}}}.
\end{equation}
根据式(2.1),当$R_{11}/R_{01}\geq\sqrt{2}$时,二维模式晶体生长形态仅为\{01\}单形;当$R_{11}/R_{01}\leq\sqrt{2}/2$时,二维模式晶体生长形态仅为\{11\}单形;当$\sqrt{2}/2<R_{11}/R_{01}<\sqrt{2}$时,二维模式晶体生长形态为\{01\}与\{11\}两种单形所组成的聚形。

同理,对于\{001\}与\{111\}两种单形所组成的立方晶系晶体形态,不难证明,它取决于(001)与(111)两晶面的相对生长速率的比值。