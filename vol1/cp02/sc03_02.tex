\subsection{生长界面稳定性的判据}
生长晶体时为了使界面稳定,寻求界面稳定性的判据显然是很重要的,确定界面是否稳定,可通过界面附近熔体的温度梯度、溶液中溶\footnote{原文作“熔”。}质的浓度梯度和界面效应等途径来做出判断的。

\paragraph{(1)熔体的温度梯度}晶体生长界面前沿的温度梯度,不外乎三种情况,一种是正温度梯度,即$(dT_l/dx)>0$,$x$的方向指向熔体,这样的熔体称为过热熔体。另一种情况称为负温度梯度,即$(dT_l/dx)<0$,这样的熔体称为过冷熔体。第三种情况是不常见的,即界面前沿的温度为熔体熔点温度,这时$\dfrac{dT_l}{dx}=0$。

对于过热熔体,如果平坦界面在偶然的外界因素干扰下而出现了凹凸不平,但由于离开界面的熔体温度梯度为正值,界面的凸起部位必然处于较高的温度$T_l$。由于$T_l>T_0$($T_0$为熔体的凝固点温度),在这种情况下,界面凸起部分的生长速率便会逐渐降低,从而被界面凹入部位的生长所追及,最后生长界面必然恢复到原来的光滑面的状态,因此这种类型的界面是稳定的,这样,熔体中的正温度梯度是有利于界面稳定性的因素,并可作为生长界面稳定性的判据。根据界面上能量(热)守恒原则
$$\kappa_s\frac{\partial T_s}{\partial x}=\kappa_l\frac{\partial T_l}{\partial x}+f\rho L$$
由于$\dfrac{\partial T_l}{\partial x}>0$,可用生长速率$f$的大小来作为界面稳定性的判据
\begin{equation}
f<\frac{\kappa_s}{L\rho}\frac{\partial T_s}{\partial x},
\end{equation}
式中$\rho$为晶体密度,$L$为结晶潜热。

对于过冷的熔体,熔体中的温度梯度为负值,即$(\partial T_l/\partial x)<0$,这时熔体中的温度$T_l$低于熔点温度$T_0$,在这种情况下,有利于平坦界面受外界因素干扰而产生的凸起部位的生长,这种类型的界面显然是不稳定的,因此熔体中的负温度梯度是不利于界面稳定性的因素。

当熔体中的温度等于熔点温度时,即$(\partial T_l/\partial x)=0$,整个熔体温度时均匀分布,在这种情况下,平坦界面是否稳定,那就要看平坦界面所受外界干扰大小而定,当干扰大时,平坦界面也能变为不稳定的。

