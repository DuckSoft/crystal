\subsection{生长界面稳定性的判据}
生长晶体时为了使界面稳定,寻求界面稳定性的判据显然是很重要的,确定界面是否稳定,可通过界面附近熔体的温度梯度、溶液中溶\footnote{原文作“熔”。}质的浓度梯度和界面效应等途径来做出判断的。

\paragraph{(1)熔体的温度梯度}晶体生长界面前沿的温度梯度,不外乎三种情况,一种是正温度梯度,即$(dT_l/dx)>0$,$x$的方向指向熔体,这样的熔体称为过热熔体。另一种情况称为负温度梯度,即$(dT_l/dx)<0$,这样的熔体称为过冷熔体。第三种情况是不常见的,即界面前沿的温度为熔体熔点温度,这时$\dfrac{dT_l}{dx}=0$。

对于过热熔体,如果平坦界面在偶然的外界因素干扰下而出现了凹凸不平,但由于离开界面的熔体温度梯度为正值,界面的凸起部位必然处于较高的温度$T_l$。由于$T_l>T_0$($T_0$为熔体的凝固点温度),在这种情况下,界面凸起部分的生长速率便会逐渐降低,从而被界面凹入部位的生长所追及,最后生长界面必然恢复到原来的光滑面的状态,因此这种类型的界面是稳定的,这样,熔体中的正温度梯度是有利于界面稳定性的因素,并可作为生长界面稳定性的判据。根据界面上能量(热)守恒原则
$$\kappa_s\frac{\partial T_s}{\partial x}=\kappa_l\frac{\partial T_l}{\partial x}+f\rho L$$
由于$\dfrac{\partial T_l}{\partial x}>0$,可用生长速率$f$的大小来作为界面稳定性的判据
\begin{equation}
f<\frac{\kappa_s}{L\rho}\frac{\partial T_s}{\partial x},
\end{equation}
式中$\rho$为晶体密度,$L$为结晶潜热。

对于过冷的熔体,熔体中的温度梯度为负值,即$(\partial T_l/\partial x)<0$,这时熔体中的温度$T_l$低于熔点温度$T_0$,在这种情况下,有利于平坦界面受外界因素干扰而产生的凸起部位的生长,这种类型的界面显然是不稳定的,因此熔体中的负温度梯度是不利于界面稳定性的因素。

当熔体中的温度等于熔点温度时,即$(\partial T_l/\partial x)=0$,整个熔体温度时均匀分布,在这种情况下,平坦界面是否稳定,那就要看平坦界面所受外界干扰大小而定,当干扰大时,平坦界面也能变为不稳定的。

\paragraph{(2)溶质浓度梯度}如前所述,当界面前沿的熔体为正温度梯度,即$(\partial T_l/\partial x)>0$时,界面稳定,但这只限于对纯熔体而言的,而完全纯的熔体实际上是不存在的。如果考虑到溶质浓度梯度,即便是熔体的温度梯度为正值,平坦界面也有可能是不稳定的。

(图2.13 界面处溶质浓度的分布)

当熔体中含有平衡分凝系数$k_0<1$的溶质时,在晶体生长过程中,这些多余的溶质会在界面上不断地汇集,而形成溶质边界层$\delta_c$。当边界层愈接近界面时,其溶质浓度愈高,如图2.13所示。由于溶质在界面处浓集,致使熔体的凝固点温度降低,这时,晶体生长体系中熔体的凝固点温度分布可用下式表示:
\begin{equation}
T_l=T_0+mc_i\left[1+\left(\frac{1-k_0}{k_0}\right)\exp(-vx/D_l)\right],
\end{equation}
式中,$T_0$为纯熔体的凝固点温度,$m$为液相线斜率,$c_i$为晶体中溶质浓度,$k_0$为溶质的平衡分凝系数,$v$为晶体生长速率,$x$为界面指向熔体的运动坐标,$D_l$为溶质的扩散系数。这时在靠近界面处的熔体温度,可能发生两种不同的温度分布情况,如图2.14所示。

(图2.14 界面处熔体中温度分布)

在图2.14中,$T_mT_L$代表熔体的凝固点温度分布,$T_mT_A$代表实际熔体具有较大的正温度梯度线,$T_mT_B$代表实际熔体具有较小的正温度梯度线,$T_m$代表熔体的凝固点,$\delta_c$代表熔质边界层厚度。

从图2.14中可看出,如果熔体具有$T_mT_B$线所代表的正温度梯度,在溶质边界层$\delta_c$内,熔体的实际温度比熔体应具有的凝固点温度低,这样在界面附近便形成了过冷区(阴影线区),在这个区域中的熔体处于过冷状态,这种过冷区不是由于熔体中的负温度梯度产生的,而是因为溶质在界面附近的熔体中浓集而引起的,故称此为组分过冷。在界面上的实际温度与熔体的凝固点温度相等,而处于平衡状态,在$\delta_c$外其余部位的熔体则处于过热状态。

从图2.14中还可看出,在界面附近的过冷区内,随着$x$的增加过冷度亦增大,在这种情况下,当界面上出现的任何干扰,都会在过冷区内迅速地长大,这样一来,原来为光滑平坦界面就会变为凸凹不平的胞状的不稳定界面。如果熔体具有$T_mT_A$线所代表的较大正温度梯度时,在$T_mT_A$线上任何一点的温度都高于熔体应有的温度,因此,界面前沿均不会出现组分过冷现象,界面仍可保持稳定。

根据实际的熔体正温度梯度($T_mT_A$线和熔体的应有的平衡温度线($T_mT_L$)在界面上相切的条件,即实际的温度梯度线同平衡温度线具有相同的斜率,便可求出熔体不产生组分过冷的临界条件。这个实际的熔体温度梯度($G_l$)为溶质浓度曲线的斜率$dc_l/dx$与相图的液相线的斜率$m$的乘积。

求解界面处一维扩散方程,即可得到溶质浓度曲线的斜率$dc_l/dx$值
$$D_l\frac{\partial c_l}{\partial x}=(c_{l(0)}-c_s)v,$$
\begin{equation}
\frac{\partial c_l}{\partial x}=[c_{l(0)}-c_s]\cdot\frac{v}{D_L}.
\end{equation}
但是
$$c_s=k^*c_{l(0)}=k_ec_l,$$
所以
$$\frac{\partial c_l}{\partial x}=c_l(l-k^*)v\frac{k_e}{k^*D_l},$$
式中,$c_l$为熔体整体浓度,$k^*$为界面溶质分凝系数,$k_e$为有效分凝系数,$v$为晶体生长速率。因此,不产生熔体组分过冷的临界条件为
\begin{equation}
\frac{G_l}{v}\geq\frac{mc_l(1-k^*)k_e}{k^*D_l}.
\end{equation}
从式(2.41)中可看出,为了克服组分过冷,至少应当注意三种因素的影响,即荣体重的温度梯度($G_l$)、晶体的生长速率($v$)和溶质浓度($c_l$)的影响,以便有利于保持界面的稳定性。

