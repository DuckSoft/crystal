\subsection{环境相对晶体形态影响}
同一品种的晶体,由于晶体生长环境相的不同,往往会出现不同的晶体形态。现以自由生长体系为例来说明生长条件不同对晶体生长形态的影响。

\paragraph{(1)溶剂的影响}在过去漫长的年代里,人们对晶体形态的研究,大多花在矿物晶体和人工无机化合物晶体上,近20多年来,由于对有机非线性光学晶体的研究受到了重视,从而也开始注意到对有机晶体的生长形态的研究。现仅以{\CJKsetecglue{} 3-甲基-4-硝基吡啶-1-氧}晶体,(简称POM晶体),为例来说明溶剂对晶体形态的影响。

溶液法是研制块状有机非线性光学晶体的主要方法之一。与无机晶体水溶液生长不同之点是,有机晶体可选择多种不同有机非水溶剂。溶液中溶质与溶剂之间的相互作用对晶体生长过程有着极为重要的影响,因此,可从分子水平的晶体微观结构来研究{\CJKsetecglue{}溶质-溶剂}间的相互作用,研究晶体生长的基元化过程与晶体生长的脱溶剂化过程,进而研究晶体的生长机制与生长动力学规律,这些研究对有机晶体生长理论的发展和实际应用均有重要意义。

POM晶体时硝基吡啶类有机分子晶体,它比其他芳香硝基化合物的紫外截止波长更短,并具有较大的倍频系数,是研制非线性光学器件的候选材料。POM可溶于水、乙醇、甲苯、苯、丙酮、乙腈、环己酮、二甲基甲酰胺、二甲亚砜等溶剂中,因此,要想生长出优质大尺寸的POM晶体,优选最佳化的溶剂是一个很重要的生长条件。

溶质与溶剂间相互作用不仅影响溶液的溶解度,而且对晶体的生长形态会产生很大的影响。从环己酮、二甲基甲酰胺和二甲亚砜三种不同溶剂中生长出来的POM,其晶体形态如图2.6所示。

(图2.6)

从图2.6中可看出,POM晶体的生长形态主要是由$\{100\},\{210\},\{302\}$等单形所组成。之所以能出现不同的生长形态,可能是由于溶剂分子与某一晶面上溶质分子具有较强的选择吸附作用,难于脱溶剂化,从而降低了该晶面的生长速率,其结果便引起了晶体生长形态的变化。

\paragraph{(2)溶液pH值的影响}晶体从水溶液中生长的一个显著特点就是溶液pH值的变化对晶体生长形态有影响,控制溶液pH值的大小也是生长优质完整单晶的一个重要条件,现以$\rm\alpha-LiIO_3$晶体生长为例来加以说明。$\rm\alpha-LiIO_3$晶体是一种较典型的极性晶体,这种晶体在极轴两端的生长速率有明显的差异,在溶液的pH值为2.5、蒸发温度为70℃的生长条件下,快生长端生长速率$v_{[0001]}$是慢生长端生长速率$v_{[000\bar{1}]}$的2倍。但有的结果是$v_{[000\bar{1}]}$方向的生长速率是$v_{[0001]}$方向的3倍。因此,发生生长速率在正、负极方向反转的情况与溶液的pH值大小是有关的,在强酸溶液条件下,$\rm\alpha-LiIO_3$晶体的生长习性与中性溶液条件时是有明显的区别,在中性溶液中生长的$\rm\alpha-LiIO_3$晶体形态如图2.7所示。

但当溶液的pH值从高于pH临界值($\rm pH_c$)改变到低于$\rm pH_c$时,晶体生长的快慢端面发生倒转。

在$\rm pH<pH_c$的水溶液中生长$\rm\alpha-LiIO_3$的晶体形态,如图2.8所示。

(图2.7)(图2.8)

此时晶体的生长速率:$v_{[0001]}<v_{[000\bar{1}]}$,晶体形态是由六角锥面$\{10\bar{1}1\}$和$\{21\bar{3}2\}$以及六方柱面$\{10\bar{1}0\}$单形所组合的聚形。在$[0001]$端面方向有12个面,6个$\{10\bar{1}1\}$面呈五边形;6个$\{21\bar{3}2\}$呈四边形。在$[000\bar{1}]$端面方向,则仍由6个$\{10\bar{1}\bar{1}\}$面围成,每个面均呈三角形。

关于$\rm\alpha-LiIO_3$晶体的生长机制,有人曾提出过一些理论。诸如用离子电荷平衡以及$\rm Li^+$和$\rm IO_3^-$热骚动的差别来解释$\rm\alpha-LiIO_3$晶体生长速率的各向异性;从界面分子组浓度看$\rm\alpha-LiIO_3$晶体的机型生长;用Zata电势变号以及$\rm I_2$分子吸附的观点来解释$\rm\alpha-LiIO_3$晶体生长过程等理论。
