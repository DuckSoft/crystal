\section{晶体生长的输运过程}
晶体生长包括一系列过程,诸如晶体生长基元形成过程、晶体生长的输运过程、晶体生长界面动力学过程等,其中输运过程是一个重要的环节。从宏观的观点来看,晶体生长过程实际上是一个热量、质量和动量的输运过程。晶体生长是空间不连续与非均匀化的过程,结晶作用仅在生长界面上发生。对熔体生长来说,界面的推进是通过温度梯度进行的。要想生长出优质完整的晶体,在熔体生长体系中,必须建立合理的温度梯度分布,因为它和热量输运中的热传输机制相联系,晶体生长时所释放出的结晶潜热必须及时地从界面处输运出去,然后才能发生凝固过程。晶体从气相或溶液中生长时,生长基元首先从生长体系的其他部位输运到生长界面,然后再进行生长基元脱溶剂化、脱附、表面扩散、生长基元进入生长位置等界面反应过程。

晶体的生长输运过程主要包括热量、质量和动量输运等。这些效应均称为输运效应,而各种不同形式的输运效应是相互联系的,例如气相各部分的运动速度可以彼此不同而造成动量输运,但在气相动量运输的同时会引起质量的输运;同样,在气相中,由于温度梯度而造成的热量输运也会引起质量输运效应。

晶体生长的输运过程对其生长速率产生限制作用,并支配着生长界面的稳定性,对于生长晶体的质量有着极其重要的作用。
晶体生长的输运理论是流体力学理论发展的一个重要组成部分,多年来一直是人们所关注的研究课题,并做了不少有意义的理论与实验工作,然而有关这一方面定量的研究结果至今还不多,造成这种状况的原因是多方面的,例如对粘滞性液体动力学方程求解的困难,特别是对于一般晶体生长过程中的实际边界条件,方程式的求解基本上是不大可能的;其次,由于受某些晶体生长条件的限制(高温、高压、密封、液体不透明等),使直接观察与测量造成困难,这也是影响深入研究的原因。

\subsection{输运的类型}
 %输运的类型
\subsection{边界层理论}
 %边界层理论
