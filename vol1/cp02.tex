\chapter{晶体生长动力学}
\authors{张克从\quad 陈万春}

晶体生长是一种相变过程,它是各种复杂物理现象相互作用的结果,它受晶体生长热力学和动力学等各种因素相互作用的影响,加之人们对熔体或溶液的结构细节仍缺少充分认识,因此,至今仍难为实际晶体生长过程提出一个完备的理论模型,致使晶体生长理论与其实践仍存有一定的差距。

概括地讲,晶体生长理论应包括热力学和动力学两大方面。本书第一张主要阐明了晶体生长热力学理论,本章专门论述晶体生长动力学。

晶体生长动力学主要是阐明在不同生长条件下的晶体生长机制,以及晶体生长速率与生长驱动力间的规律。晶体生长界面结构决定了生长机制,不同的生长机制表现出不同的生长动力学规律。晶体生长速率受生长驱动力的支配,当改变生长介质的热量或质量输运时,晶体生长速率也随之而改变。晶体生长形态取决于晶体的各晶面间的相对生长速率。当生长介质的输运性质以及其他动力学因素改变时,不仅能使晶体生长速率发生变化,而且会影响到晶体生长形态与生长界面的稳定性,晶体生长界面是否稳定,关系到生长单晶的完整性。因此,晶体的完整性与其生长动力学有着密切的联系。

\section{晶体生长形态}
人们自17世纪初就开始了对矿物晶体和天然结晶的形态研究,几个世纪以来,通过对晶体形态的反复观察、分析与比较,归纳出一系列有关晶体形态(几何)的经验定律,诸如晶面角守恒定律、有理指数定律、晶体对称性定律、晶带定律等,这些经验点那个了吧为后来的晶体学发展奠定了学科基础。晶体生长形态是其内部结构的外在反映,晶体的各个晶面间的相对生长速率决定了他的生长形态。晶体生长形态虽受其内部结构的对称性、结构基元间键合和晶体缺陷等因素的制约,但在很大程度上还受到生长环境相的影响。因此,同一品种的晶体(即成分与结构均相同),既能形成具有对称特征的几何多面体,又能生长成特殊的形态。晶体生长形态能部分地反应出它的形成历史,因此,研究晶体生长形态,有助于人们认识晶体生长动力学过程,可为探讨实际晶体生长机制提供线索。

广义来讲,对于晶体生长形态学,不仅要研究晶体生长后的宏观几何外形以及其生长过程中宏观几何外形的演变,而且还应包括生长界面的显微形态和生长界面的稳定性等内容,这些研究内容,我们在下面将进行分析与讨论。

\subsection{晶体生长形态与生长速率间的联系}
一般说来,晶体在自由的生长体系中生长,晶体的各晶面生长速率是不同的,即晶体的生长速率是各向异性的。通常所说的晶体的晶面生长速率$R$是指在单位时间内晶面$(hkl)$沿其法线方向向外平行推移的距离($d$),并称为线性生长速率。

晶体生长的驱动力来源于生长环境相(汽相、溶液、熔体)的过饱和度($\Delta c$),或过冷度($\Delta T$)。人工生长单晶时,在保证晶体生长质量的前提下,人们总是希望提高生长速率,但是要维持恒定的生长速率,其工艺技术是很难达到的。常常由于晶体生长速率的改变,导致晶体缺陷的产生,这不仅有损于晶体的完整性,而且晶体的生长形态也要发生变化的。

晶体生长形态的变化来源于个镜面相对生长速率(比值)的改变,现以二维模式晶体生长为例来说明晶面的相对生长速率的变化与晶体生长形态间的联系(如图2.1所示)。

(图2.1)

在图2.1中,$l_{11},l_{01}$分别代表(11),(01)晶面的大小,$R_{11},R_{01}$分别代表(11),(01)晶面的生长速率。从图2.1所表明的简单的几何关系中可求得
\begin{align*}
R_{01}&=\frac{l_{01}}{2}+\sqrt{2}\cdot\frac{l_{11}}{2},\\
\frac{l_{01}}{2}&=\sqrt{2}\cdot R_{11}-R_{01},\\
R_{11}&=\sqrt{2}\cdot R_{01}-\frac{l_{11}}{2},\\
\frac{l_{11}}{2}&=\sqrt{2}\cdot R_{01}-R_{11}
\end{align*}
\begin{equation}
\frac{l_{01}}{l_{11}}=\frac{\displaystyle\sqrt{2}\left(\frac{R_{11}}{R_{01}}\right)-1}{\displaystyle\sqrt{2}-\frac{R_{11}}{R_{01}}}.
\end{equation}
根据式(2.1),当$R_{11}/R_{01}\geq\sqrt{2}$时,二维模式晶体生长形态仅为\{01\}单形;当$R_{11}/R_{01}\leq\sqrt{2}/2$时,二维模式晶体生长形态仅为\{11\}单形;当$\sqrt{2}/2<R_{11}/R_{01}<\sqrt{2}$时,二维模式晶体生长形态为\{01\}与\{11\}两种单形所组成的聚形。

同理,对于\{001\}与\{111\}两种单形所组成的立方晶系晶体形态,不难证明,它取决于(001)与(111)两晶面的相对生长速率的比值。 %晶体生长形态与生长速率间的联系
\subsection{晶体生长的理想形态}
具有几何形态的实际晶体,经过晶面角测量和极射赤平投影后,能够去伪存真,而描绘出晶体的理想形态。晶体的理想形态可分为单形和聚形。

当晶体在自由体系中生长时,若生长出的晶体形态的各个晶面的面网结构相同,而且各个晶面都是同形等大,这样的晶体理想形态称为单形。若在晶体的理想形态中,具有两套以上不同形、也不等大的晶面,这种晶体的理想形态称为聚形,而聚形是由数种单形构成的。

研究实际晶体生长形态,首先应当研究它的理想形态,以寻求晶面或晶带在三维空间分布的几何规律性,然后再进一步研究晶体生长形态出现的外在原因。 %晶体生长的理想形态
\subsection{晶体生长的实际形态}
 %晶体生长的实际形态
\subsection{晶体几何形态与其内部结构间的联系}
 %晶体几何形态与其内部结构间的联系
\subsection{环境相对晶体形态影响}
同一品种的晶体,由于晶体生长环境相的不同,往往会出现不同的晶体形态。现以自由生长体系为例来说明生长条件不同对晶体生长形态的影响。

\paragraph{(1)溶剂的影响}在过去漫长的年代里,人们对晶体形态的研究,大多花在矿物晶体和人工无机化合物晶体上,近20多年来,由于对有机非线性光学晶体的研究受到了重视,从而也开始注意到对有机晶体的生长形态的研究。现仅以{\CJKsetecglue{} 3-甲基-4-硝基吡啶-1-氧}晶体,(简称POM晶体),为例来说明溶剂对晶体形态的影响。

溶液法是研制块状有机非线性光学晶体的主要方法之一。与无机晶体水溶液生长不同之点是,有机晶体可选择多种不同有机非水溶剂。溶液中溶质与溶剂之间的相互作用对晶体生长过程有着极为重要的影响,因此,可从分子水平的晶体微观结构来研究{\CJKsetecglue{}溶质-溶剂}间的相互作用,研究晶体生长的基元化过程与晶体生长的脱溶剂化过程,进而研究晶体的生长机制与生长动力学规律,这些研究对有机晶体生长理论的发展和实际应用均有重要意义。

POM晶体时硝基吡啶类有机分子晶体,它比其他芳香硝基化合物的紫外截止波长更短,并具有较大的倍频系数,是研制非线性光学器件的候选材料。POM可溶于水、乙醇、甲苯、苯、丙酮、乙腈、环己酮、二甲基甲酰胺、二甲亚砜等溶剂中,因此,要想生长出优质大尺寸的POM晶体,优选最佳化的溶剂是一个很重要的生长条件。

溶质与溶剂间相互作用不仅影响溶液的溶解度,而且对晶体的生长形态会产生很大的影响。从环己酮、二甲基甲酰胺和二甲亚砜三种不同溶剂中生长出来的POM,其晶体形态如图2.6所示。

(图2.6)

从图2.6中可看出,POM晶体的生长形态主要是由$\{100\},\{210\},\{302\}$等单形所组成。之所以能出现不同的生长形态,可能是由于溶剂分子与某一晶面上溶质分子具有较强的选择吸附作用,难于脱溶剂化,从而降低了该晶面的生长速率,其结果便引起了晶体生长形态的变化。

\paragraph{(2)溶液pH值的影响}晶体从水溶液中生长的一个显著特点就是溶液pH值的变化对晶体生长形态有影响,控制溶液pH值的大小也是生长优质完整单晶的一个重要条件,现以$\rm\alpha-LiIO_3$晶体生长为例来加以说明。$\rm\alpha-LiIO_3$晶体是一种较典型的极性晶体,这种晶体在极轴两端的生长速率有明显的差异,在溶液的pH值为2.5、蒸发温度为70℃的生长条件下,快生长端生长速率$v_{[0001]}$是慢生长端生长速率$v_{[000\bar{1}]}$的2倍。但有的结果是$v_{[000\bar{1}]}$方向的生长速率是$v_{[0001]}$方向的3倍。因此,发生生长速率在正、负极方向反转的情况与溶液的pH值大小是有关的,在强酸溶液条件下,$\rm\alpha-LiIO_3$晶体的生长习性与中性溶液条件时是有明显的区别,在中性溶液中生长的$\rm\alpha-LiIO_3$晶体形态如图2.7所示。

但当溶液的pH值从高于pH临界值($\rm pH_c$)改变到低于$\rm pH_c$时,晶体生长的快慢端面发生倒转。

在$\rm pH<pH_c$的水溶液中生长$\rm\alpha-LiIO_3$的晶体形态,如图2.8所示。

(图2.7)(图2.8)

此时晶体的生长速率:$v_{[0001]}<v_{[000\bar{1}]}$,晶体形态是由六角锥面$\{10\bar{1}1\}$和$\{21\bar{3}2\}$以及六方柱面$\{10\bar{1}0\}$单形所组合的聚形。在$[0001]$端面方向有12个面,6个$\{10\bar{1}1\}$面呈五边形;6个$\{21\bar{3}2\}$呈四边形。在$[000\bar{1}]$端面方向,则仍由6个$\{10\bar{1}\bar{1}\}$面围成,每个面均呈三角形。

关于$\rm\alpha-LiIO_3$晶体的生长机制,有人曾提出过一些理论。诸如用离子电荷平衡以及$\rm Li^+$和$\rm IO_3^-$热骚动的差别来解释$\rm\alpha-LiIO_3$晶体生长速率的各向异性;从界面分子组浓度看$\rm\alpha-LiIO_3$晶体的机型生长;用Zata电势变号以及$\rm I_2$分子吸附的观点来解释$\rm\alpha-LiIO_3$晶体生长过程等理论。
 %环境相对晶体形态影响

\section{晶体生长的输运过程}
晶体生长包括一系列过程,诸如晶体生长基元形成过程、晶体生长的输运过程、晶体生长界面动力学过程等,其中输运过程是一个重要的环节。从宏观的观点来看,晶体生长过程实际上是一个热量、质量和动量的输运过程。晶体生长是空间不连续与非均匀化的过程,结晶作用仅在生长界面上发生。对熔体生长来说,界面的推进是通过温度梯度进行的。要想生长出优质完整的晶体,在熔体生长体系中,必须建立合理的温度梯度分布,因为它和热量输运中的热传输机制相联系,晶体生长时所释放出的结晶潜热必须及时地从界面处输运出去,然后才能发生凝固过程。晶体从气相或溶液中生长时,生长基元首先从生长体系的其他部位输运到生长界面,然后再进行生长基元脱溶剂化、脱附、表面扩散、生长基元进入生长位置等界面反应过程。

晶体的生长输运过程主要包括热量、质量和动量输运等。这些效应均称为输运效应,而各种不同形式的输运效应是相互联系的,例如气相各部分的运动速度可以彼此不同而造成动量输运,但在气相动量运输的同时会引起质量的输运;同样,在气相中,由于温度梯度而造成的热量输运也会引起质量输运效应。

晶体生长的输运过程对其生长速率产生限制作用,并支配着生长界面的稳定性,对于生长晶体的质量有着极其重要的作用。
晶体生长的输运理论是流体力学理论发展的一个重要组成部分,多年来一直是人们所关注的研究课题,并做了不少有意义的理论与实验工作,然而有关这一方面定量的研究结果至今还不多,造成这种状况的原因是多方面的,例如对粘滞性液体动力学方程求解的困难,特别是对于一般晶体生长过程中的实际边界条件,方程式的求解基本上是不大可能的;其次,由于受某些晶体生长条件的限制(高温、高压、密封、液体不透明等),使直接观察与测量造成困难,这也是影响深入研究的原因。

\subsection{输运的类型}
 %输运的类型
\subsection{边界层理论}
 %边界层理论

\section{晶体生长界面的稳定性}

\subsection{研究界面稳定性应遵循的几个原则}
 %研究界面稳定性应遵循的几个原则
\subsection{生长界面稳定性的判据}
生长晶体时为了使界面稳定,寻求界面稳定性的判据显然是很重要的,确定界面是否稳定,可通过界面附近熔体的温度梯度、溶液中溶\footnote{原文作“熔”。}质的浓度梯度和界面效应等途径来做出判断的。

\paragraph{(1)熔体的温度梯度}晶体生长界面前沿的温度梯度,不外乎三种情况,一种是正温度梯度,即$(dT_l/dx)>0$,$x$的方向指向熔体,这样的熔体称为过热熔体。另一种情况称为负温度梯度,即$(dT_l/dx)<0$,这样的熔体称为过冷熔体。第三种情况是不常见的,即界面前沿的温度为熔体熔点温度,这时$\dfrac{dT_l}{dx}=0$。

对于过热熔体,如果平坦界面在偶然的外界因素干扰下而出现了凹凸不平,但由于离开界面的熔体温度梯度为正值,界面的凸起部位必然处于较高的温度$T_l$。由于$T_l>T_0$($T_0$为熔体的凝固点温度),在这种情况下,界面凸起部分的生长速率便会逐渐降低,从而被界面凹入部位的生长所追及,最后生长界面必然恢复到原来的光滑面的状态,因此这种类型的界面是稳定的,这样,熔体中的正温度梯度是有利于界面稳定性的因素,并可作为生长界面稳定性的判据。根据界面上能量(热)守恒原则
$$\kappa_s\frac{\partial T_s}{\partial x}=\kappa_l\frac{\partial T_l}{\partial x}+f\rho L$$
由于$\dfrac{\partial T_l}{\partial x}>0$,可用生长速率$f$的大小来作为界面稳定性的判据
\begin{equation}
f<\frac{\kappa_s}{L\rho}\frac{\partial T_s}{\partial x},
\end{equation}
式中$\rho$为晶体密度,$L$为结晶潜热。

对于过冷的熔体,熔体中的温度梯度为负值,即$(\partial T_l/\partial x)<0$,这时熔体中的温度$T_l$低于熔点温度$T_0$,在这种情况下,有利于平坦界面受外界因素干扰而产生的凸起部位的生长,这种类型的界面显然是不稳定的,因此熔体中的负温度梯度是不利于界面稳定性的因素。

当熔体中的温度等于熔点温度时,即$(\partial T_l/\partial x)=0$,整个熔体温度时均匀分布,在这种情况下,平坦界面是否稳定,那就要看平坦界面所受外界干扰大小而定,当干扰大时,平坦界面也能变为不稳定的。

 %生长界面稳定性的判据
\subsection{界面稳定性动力学理论}
早在1964年,Mullins和Sekerka运用干扰技术检验了球形界面和平坦界面的稳定性,从而奠定了界面稳定性理论的基础。几十年来,界面稳定性动力学理论一直是一个重要的研究领域。

检验界面稳定性的干扰技术,其基本思想如下:由扩散方程求得的运动界面的运动方程,叠加一个微小的形态干扰函数,这样就得到了受干扰后的运动界面方程。人们可根据干扰后的界面满足于扩散方程这一点来确定形态干扰函数。如果形态干扰函数的振幅确实随时间而变化,那么原来的界面就是不稳定的,否则就是稳定的,这就是检验界面稳定性的所谓干扰技术。

晶体生长过程中,移动的界面上不可避免的会出现干扰,界面上一旦出现几何干扰,也必然在界面附近引起局部温度场和浓度场的变化,这些可相应地理解为温度干扰和浓度干扰。干扰技术的实质在于研究热量或质量在扩散场中的干扰行为,即研究干扰振幅与时间的依赖关系。界面上出现的任何周期性干扰都可用正弦函数的傅里叶级数来表示,因此,我们应该考虑到所有可能波长的正弦干扰行为。例如在强制生长体系中,当界面未受到干扰时,它恒为等速运动的平面,在运动坐标系中,其界面方程为$Z\equiv 0$。当受到正弦式的几何干扰后,界面干扰形状可用下式表示:
\begin{equation}
Z(x,t)=\phi(x,t)=\delta(t)\sin(\omega x),
\end{equation}
式中$\delta(t)$为微干扰的振幅,$\omega$为微干扰的空间频率,$\lambda=\frac{2\pi}{\omega}$为微干扰的波长。通常,经过较长的时间后,微干扰振幅$\delta(t)$与时间的关系可表示为指数关系,即
\begin{equation}
\delta(t)\sim\text{常数}\cdot\exp(pt),
\end{equation}
于是有
\begin{equation}
p=\lim\frac{d\delta(t)}{dt}\cdot\frac{1}{\delta(t)}=\frac{\dot{\delta}}{\delta}
\end{equation}
式中$\dot{\delta}/\delta$称为单位振幅的变化率。

由式(2.44)可知,在界面上一旦出现微干扰,而干扰取决于$p$值,当$p$为正值时,$\delta(t)$便随时间而增大,界面是不稳定的;但当$p$为负值时,$\delta(t)$随时间而减弱,界面是稳定的。从式(2.45)中可以看出,$p$值的大小决定了$\delta(t)$的增长或衰减的速率。作为界面稳定性的判据是微干扰随时间而变化。如果$\dfrac{d\delta(t)}{dt}<0$,则界面是稳定的,如果$\dfrac{d\delta(t)}{dt}>0$,则界面是不稳定的,其相应地图形如图2.16所示。

(图2.16 界面受微干扰后的变化情况)

为了研究界面的稳定性,关键在于研究扩散场中所有波长的干扰振幅与时间的关系。干扰本身的行为受到扩散场的支配,因而,干扰后的温度场和浓度场必然满足与时间相关的扩散方程式。在充分长的时间后,温度干扰和浓度干扰振幅与时间的关系,与几何干扰所得的结论一样,界面是否是稳定的,主要取决于单位振幅的变率($\dot{\delta}/\delta$)。

上述所谈的微干扰界面稳定性理论描述了一具有任意波长的无限小干扰的界面增长和衰减,这对了解界面是否稳定是十分方便的。在一组给定的生长条件下,利用上述理论便可得到干扰增长或衰减的指数速率。这组给定的生长条件包括未受干扰界面的温度梯度、浓度梯度、热扩散率、溶质扩散率、潜热、宏观生长速率、界面能等。因此,这一线性动力学理论使人们对晶体生长过程的认识大大地深刻多了。

在晶体生长过程中,微干扰出现以后,界面邢台的变化远比干扰出现时的形态变化更为复杂,例如胞状界面的出现、枝蔓晶体生长是不能用线性动力学来描述的,而有待于用非线性界面动力学予以解释。

保持界面的稳定性,对生长优质、完整的晶体至关重要。晶体生长的实验证明,通常导致界面不稳定的因素主要是以下几个方面:(1)过小的温度梯度;(2)太快的生长速率;(3)太多的溶质(对熔体生长而言)或太多的溶剂(对溶液生长而言)。欲促使界面稳定,其补救的方法如下:可利用较大的温度梯度、较慢的生长速率、较小的溶质浓度(对熔体生长而言)或较浓的溶液来生长晶体,并要综合各种影响因素加以处理。 %界面稳定性动力学理论

\clearpage
\section{晶体生长界面结构理论模型}

\subsection{完整光滑面理论模型}
 %完整光滑面理论模型
\subsection{非完整光滑面理论模型}
 %非完整光滑面理论模型
\subsection{粗糙界面理论模型}
 %粗糙界面理论模型
\subsection{扩散界面理论模型}
 %扩散界面理论模型


\section{晶体生长界面动力学}

\subsection{完整光滑面的生长}
 %完整光滑面的生长
\subsection{非完整光滑面的生长}
 %非完整光滑面的生长
\subsection{粗糙界面的生长}
 %粗糙界面的生长
\subsection{扩散界面的生长}
 %扩散界面的生长
\subsection{BCF体扩散理论和GGC体-表面耦合扩散理论}
 %BCF体扩散理论和GGC体-表面耦合扩散理论
\subsection{高聚物晶体生长}
 %高聚物晶体生长

\section{晶体生长动力学研究实验方法}

\subsection{等组分方法}
 %等组分方法
\subsection{原位实时观察法}
 %原位实时观察法
\subsection{显微照相法}
 %显微照相法
\subsection{光散射法}
 %光散射法
\subsection{计算机模拟法}
 %计算机模拟法

\clearpage
\section{空间晶体生长的微重力效应}

\subsection{微重力概念}
宇宙空间任何两个物体之间均存在吸引力,而且其引力大小正比于两物体的质量之积,反比于它们之间的距离,这就是著名的牛顿万有引力定律。自由落体因受到地心的引力,会产生铅直方向的加速运动。若把地球表面的平均重力加速度记为$g_0$,则$g_0=980 \rm\ cm/s^2$。在太空中的物体,由于与地球作用距离的增加,重力加速度将减少。若重力减少到零,则物体处于零重力失重状态。所谓微重力是指重力减少到地球表面重力百万分之一时的重力场。严格说来,在数值上,微重力的精确定义应为$g=10^{-6}g_0$,但目前“微重力”概念已延拓,通常把$g<10^{-2}g_0$的重力环境均称为微重力环境。考虑到未来的星际飞行,对于微重力的概念作些延拓是有意义的。因为月球上的重力场为$0.16g_0$,火星上的重力场为$0.3g_0$,微重力概念延拓以后,就可以把星际飞行重力环境中的科学实验称为微重力科学实验。

如何获得微重力环境,开展微重力科学研究呢?从理论上说,根据牛顿万有引力定律,只要把实验舱发射到远离地球的太空,就可以实现微重力环境。可是,由于技术上的原因,目前并不走这条路,而是根据爱因斯坦等效性原理模拟失重状态。爱因斯坦等效原理叙述为:自由落体的重力被惯性局部抵偿。因而自由飞行的轨道航天器作为自由落体的特殊形式,可以模拟微重力或零重力环境。在自由漂移的航天器内,其质心位置上的重力与离心力相平衡,处于失重状态。

所有类型的自由下落轨迹都适用于微重力环境的模拟。目前用于模拟微重力环境的设备和装置见表2.5。

(表2.5)

在分子水平上,重力是弱作用力,它比氢原子中静电力小$10^{40}$倍,比氢分子中电磁力小$10^35$倍。因此,由分子间作用力所决定的性质估计和重力无关。但若没有其他作用力与重力平衡,则在存在大质量时,重力会起主要作用。那么重力究竟在哪些方面起重要影响呢?美国微重力研究委员会1992年公布了到2000年的微重力研究战略,其中提到了下述7个方面:

\paragraph{(1)重力作为流体中对流的驱动力}因温度和组分的不均匀而产生的密度差,使静止状态的流体发生对流,导致热量传递和质量输运。在地面上原子或分子的对流比因布朗运动引起的漫迁移大1个数量级。
\paragraph{(2)重力作为相分离的驱动力}对于某些热力学体系,存在共存相,其化学势等同,但密度有差别,在重力作用下,经过长时间后会发生沉淀,导致相分离。
\paragraph{(3)重力作为决定流体自然表面形态的力}在热力学平衡或近平衡系统中,流体表面或界面形态决定于力的平衡。比如气泡、液滴和浮区的形状主要受重力支配。
\paragraph{(4)临界现象}在临界点附近,流体的压缩率非常大,流体在自身重量下分层。因此,对于任何有限体积流体热力学参量的测量,实际测出的是不同密度流体的平均性质而不是临界点的真实性质。这种重力对实验所施加的基本限制,只有在微重力环境下才能得到抑制。
\paragraph{(5)存在极弱键合力的系统——生命系统}生命系统中生物大分子的键合力极弱,重力扮演重要作用。比如蛋白质晶体的结晶、生物细胞的融合等。
\paragraph{(6)存在极大质量和极长时间的系统}比如:广义相对论和引力波。
\paragraph{(7)大尺寸构件和超长距离}比如建筑和桥梁的协强,地球及其大气层内长距离自然现象等。

 %微重力概念
\subsection{空间微重力晶体生长实验}
空间晶体生长始于70年代初期的Apollo空间实验。1971---1973年期间,在Apollo 14,16和17号飞船上进行了Benard胞、对流、凝固和复合物材料晶体生长实验,为微重力条件下的晶体生长打下了最初的实验基础。自1973年美国在天空实验室首先生长出酒石酸钾钠晶体以来,世界上发达国家为争夺空间科学研究和空间开发的高技术优势,进行了多次微重力条件下晶体生长实验。同时,在地面上进行了大量实验和理论模拟。研究人员先后在空间条件下探索了从光电子材料到蛋白质生物大分子晶体等几十种功能晶体的生长,在实验设备方面已动用巨资制造了各种晶体生长实验装置,以适用双扩散法、低温溶液法、气相法、浮区法、高温移动加热器法和坩埚下降法等晶体生长方法的实验要求。在晶体生长基础理论方面,研究了微重力条件下的晶体成核理论、生长动力学理论、纯扩散生长机制和形态稳定性理论。一系列理论分析和实验表明,开展微重力条件下的晶体生长规律研究,不仅会加深对微重力效应的认识,而且可能引起材料科学的重大突破。人们预计在微重力条件下将有可能生长出极有价值的新晶体。

20多年来,空间晶体生长大体经历3个发展阶段。70年代初、中期为第一阶段,在此期间,人们误以为空间晶体生长是零重力条件下的物化反应过程。似乎在地面上存在的许多与重力有关的问题,在微重力条件下都可以得到解决。实验结果表明,重力大大减弱以后,表面张力驱动对流上升为占支配地位的作用力,为了探索微重力条件下的晶体生长规律,必须深入研究次级力场效应。70年代中期到80年代中期为第二阶段,这是晶体生长设备的设计、制造和新方法探索阶段。在这阶段内,人们利用空间微重力条件生长各种晶体,回到底面以后对其形态、结构和晶体完整性进行观察和分析。通过空间-地面的实验对比,评估微重力条件下晶体生长的潜在效益。80年代后期开始为第三阶段。将激光全息术和显微照相术应用于空间晶体生长,在微重力条件下实时记录晶体生长过程,研究空间晶体生长的特有规律,从而获得最佳微重力晶体生长条件。下面介绍几例空间晶体生长实验:

\paragraph{(1)熔体晶体生长}迄今为止,在空间微重力条件下,从熔体中生长的方法有坩埚下降法、区熔法和移动加热器法,生长实验见表2.6。

(表2.6)

\paragraph{(2)气相晶体生长}在空间环境下,应用化学气相输运和物理气相输运法研究了$\rm\alpha-HgI_2$,$\rm CdSe$,$\rm PbTe$,$\rm ZnTb$,$\rm ZnO$和$\rm Ge$等单晶体生长,其中对$\rm\alpha-HgI_2$晶体生长研究较为系统,在1985-1992年期间进行了5次空间微重力实验。最成功的一次实验是1992年进行的国际微重力实验室(IML-1)实验。据报道,在空间所获得的$\rm\alpha-HgI_2$单晶体的尺寸和性能都很理想。表2.7是几次$\rm\alpha-HgI_2$空间实验条件和结果。

\paragraph{(3)高温溶液晶体生长}空间高温溶液晶体生长中所应用的技术是坩埚下降法和移动加热器法。表2.8列述了某些晶体的生长参量。

(表2.7)

(表2.8)

\paragraph{(4)溶液法晶体生长}对于高溶解度晶体材料,在地面上一般采用恒温蒸发法、降温法和循环流动法。原则上在空间环境也可以采用上述方法。目前,由于技术上的原因,一般采用降温法。对于低溶解度晶体材料来说,在地面上采用凝胶法生长。凝胶的作用在于:(1)抑制自然对流,使晶体生长过程受纯扩散的支配;(2)支撑晶体,使晶体悬于容器中免受外力的作用。凝胶的这两个作用在微重力条件下都得到满足。所以在空间用溶质扩散反应法就可以替代凝胶法,习惯上把它称为双扩散法。表2.9列出溶液法晶体生长的实例。

(表2.9)

 %空间微重力晶体生长实验
\subsection{微重力状态下的输运过程}
重力对流体的输运性质有重要影响。这种影响是宏观的,主要与重力驱动的对流和由阿基米德力引起的上浮和下沉有关,因而流体的所有输运性质都与微重力这个概念密切相关。

晶体生长过程是固体结晶基元和流体结晶基元在特定条件下的相互作用和能量再分配过程。结晶过程总是离不开固体成分和流体成分。因固体内部结合力远大于1个$g_0$的重力,所以结晶过程仅需考虑流体成分受重力的影响,在溶液中流体的粘着力和表面张力与$g_0$的重力同数量级,液体的常规性质是内分子力和重力相互作用的结果。假如重力消失,流体的行为将受内分子力支配。因而,微重力条件将影响晶体的生长过程。结晶体系中能量的再分配离不开质量的输运和热量的传递,质量的输运通过对流和扩散途径进行,而热量传递则通过对流传导和辐射途径进行。在微重力条件下,扩散将支配质量输运过程。晶体生长的输运问题将变成Stefan问题。在数学物理方程的处理上就变成求解自由边界和运动边界的扩散微分方程问题。基于上述原因,为了认识微重力条件下晶体生长的动力学特征,研究扩散引起的质量输运的理论和实验现象就十分必要了。

\paragraph{(1)扩散理论}扩散是气体、液体和固体中的普遍现象。扩散系数的典型值对气体是$\rm 1\ cm^2/s$,对液体是$10^{-5}\ \rm cm^2/s$,对固体通常则远低于$10^{-8}\ \rm cm^2/s$。解释扩散现象的基本定律是Fick第一定律和第二定律,如2.2.3节所述。流体中的扩散理论模型可归纳如下:

\begin{enumerate}[(i)]\itemsep -0.5ex
\item 准晶模型。假设流体中的原子是按照类似于在固体里的排列方式,原子或分子的扩散必须克服激活势垒。扩散系数和温度的函数关系,服从(Arrhenius)方程(2.125)
\begin{equation}
D=D_0\exp(-\frac{Q}{kT}),
\end{equation}
式中$D_0$是指数前因子,$Q$是激活能,$k$是玻尔兹曼常数,$T$为绝对温度。

\item 自由体积模型。假设原子或分子扩散到与其相邻的自由体积内,自由体积具有临界体积$V_c$,扩散系数$D$和温度遵循下列方程:
\begin{equation}
D=AT\exp(-\frac{BV_c}{V_f}),
\end{equation}
式中$A,B$是与材料物性有关的参数,$V_f$是液体的总体积。

\item 涨落理论。假设原子或原子集团由于热振动(布朗运动)而进入周围自由体积内,而且自由体积没有临界尺寸。扩散系数和温度满足二次函数关系
\begin{equation}
D=AT^2,
\end{equation}
式中$A$是与原子相互作用势能有关的常数。

\item 能斯脱-爱因斯坦-斯托克斯(Nernst-Einstein-Stokes)关系。假设扩散系数可以表示为驱动力和摩擦力之比,即
\begin{equation}
D=\frac{kT}{6\pi\eta r},
\end{equation}
式中$\eta$为粘滞系数,$r$是扩散原子的半径。
\end{enumerate}

在地面上,由于重力驱动对流的存在使得地面实验结果不准确,甚至还不能决定哪些模型可正确地描述其与温度的依赖关系。在空间环境中,重力驱动的自然对流大大减弱或得到消除,因此可完成一些精度足够高的实验,并可精确地测定由扩散控制的质量输运性质,进而导致关于流体中扩散和输运的新的理论概念,也会促进地面晶体生长技术的变革和生长理论的完善。科学家还语言微重力条件下将会出现新晶体和新理论。

\paragraph{(2)空间实验结果}
\begin{enumerate}[(i)]\itemsep -0.5ex
\item 自扩散实验。1983年11月在空间实验室-1中进行了$\rm Sn$的自扩散实验。用16个样品,作了不同温度下的8次实验。每次用直径分别为$\rm 1\ mm$和$\rm 3\ mm$的两个样品。每个样品长$\rm 55\ mm$,在其一端固定$\rm 4\ mm$厚的稳态$\rm Sn$同位素圆片。稳态的同位素$\rm^{124}Sn$和$^{112}Sn$用来检测自扩散系数和同位素效应。表2.10列出了$\rm Sn$的自扩散实验结果。

(表2.10)

在表2.10中$E$为同位素效应。$D_{112}$和$D_{124}$分别为两种Sn同位素的扩散系数。从空间实验获得的扩散系数比地面值低20-40\%。这表明在地面上进行$D$的测量时,对流有明显影响。另一个重要结果是扩散系数和温度的关系为二次函数,即

(图2.51)

\begin{equation}
D({\rm Sn})=AT^2,
\end{equation}
式中$A=0.745\times 10^{-16}\ \rm cm^2/k^2s$,即在$\pm2\%$的精度内,扩散系数和温度是平方关系。这个结果支持了涨落理论。

\item 互扩散实验。空间条件下的玻璃熔体实验首先是在探空火箭上进行的,$\rm Na_2O\cdot3SiO_2$和$\rm Rb_2O\cdot3SiO_2$熔体在$1200$℃温度下持续$\rm 200\ s$。后来在空间实验室(LS-1)上重做了实验。实验温度$\rm 1280$℃,持续时间$\rm 1730\ s$。图2.51示出微重力条件下的实验结果。图2.52示出相同条件的地面实验结果。这两个对比实验表明:在地面上由于有对流的存在,Na和Rb的浓度分布曲线不规则,用它几乎不可能测定有意义的扩散系数。而在微重下,理论数据和实验数据符合得很好。空间实验数据分析还表明:当Na和Rb的浓度相同时,扩散系数最大。

(图2.52)

\item GeSe和GeTe的质量输运速率。在天空实验室(Skylab-3)上已研究了IV-VI族半导体GeSe和GeTe的化学输运反应。用$\rm GeI_4$作为输运载体。密封石英管的尺寸为$\Phi\rm13\times150\ mm$。在实验过程中,先升温$\rm3\ h$,再在生长温度下恒温$\rm33\ h$,然后降温到室温,降温过程历时$\rm12\ h$。表2.11列出了空间和地基实验数据。

(表2.11)

\noindent 表2.11中,$A,B$和$C$是样品编号。试验结果表明,$\rm\mu g$下的质量输运速率依赖于材料和载体气压,其数值可以小于、大于或等于$1\rm\ g$条件下的实验值。
\end{enumerate} %微重力状态下的输运过程
\subsection{微重力环境下的晶体生长形态}
 %微重力环境下的晶体生长形态
\subsection{微重力对晶体组分、结构和完整性的影响}
 %微重力对晶体组分、结构和完整性的影响

