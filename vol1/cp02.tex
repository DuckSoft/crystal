\chapter{晶体生长动力学}
\authors{张克从\quad 陈万春}

晶体生长是一种相变过程,它是各种复杂物理现象相互作用的结果,它受晶体生长热力学和动力学等各种因素相互作用的影响,加之人们对熔体或溶液的结构细节仍缺少充分认识,因此,至今仍难为实际晶体生长过程提出一个完备的理论模型,致使晶体生长理论与其实践仍存有一定的差距。

概括地讲,晶体生长理论应包括热力学和动力学两大方面。本书第一张主要阐明了晶体生长热力学理论,本章专门论述晶体生长动力学。

晶体生长动力学主要是阐明在不同生长条件下的晶体生长机制,以及晶体生长速率与生长驱动力间的规律。晶体生长界面结构决定了生长机制,不同的生长机制表现出不同的生长动力学规律。晶体生长速率受生长驱动力的支配,当改变生长介质的热量或质量输运时,晶体生长速率也随之而改变。晶体生长形态取决于晶体的各晶面间的相对生长速率。当生长介质的输运性质以及其他动力学因素改变时,不仅能使晶体生长速率发生变化,而且会影响到晶体生长形态与生长界面的稳定性,晶体生长界面是否稳定,关系到生长单晶的完整性。因此,晶体的完整性与其生长动力学有着密切的联系。

\section{晶体生长形态}

\subsection{晶体生长形态与生长速率间的联系}
一般说来,晶体在自由的生长体系中生长,晶体的各晶面生长速率是不同的,即晶体的生长速率是各向异性的。通常所说的晶体的晶面生长速率$R$是指在单位时间内晶面$(hkl)$沿其法线方向向外平行推移的距离($d$),并称为线性生长速率。

晶体生长的驱动力来源于生长环境相(汽相、溶液、熔体)的过饱和度($\Delta c$),或过冷度($\Delta T$)。人工生长单晶时,在保证晶体生长质量的前提下,人们总是希望提高生长速率,但是要维持恒定的生长速率,其工艺技术是很难达到的。常常由于晶体生长速率的改变,导致晶体缺陷的产生,这不仅有损于晶体的完整性,而且晶体的生长形态也要发生变化的。

晶体生长形态的变化来源于个镜面相对生长速率(比值)的改变,现以二维模式晶体生长为例来说明晶面的相对生长速率的变化与晶体生长形态间的联系(如图2.1所示)。

(图2.1)

在图2.1中,$l_{11},l_{01}$分别代表(11),(01)晶面的大小,$R_{11},R_{01}$分别代表(11),(01)晶面的生长速率。从图2.1所表明的简单的几何关系中可求得
\begin{align*}
R_{01}&=\frac{l_{01}}{2}+\sqrt{2}\cdot\frac{l_{11}}{2},\\
\frac{l_{01}}{2}&=\sqrt{2}\cdot R_{11}-R_{01},\\
R_{11}&=\sqrt{2}\cdot R_{01}-\frac{l_{11}}{2},\\
\frac{l_{11}}{2}&=\sqrt{2}\cdot R_{01}-R_{11}
\end{align*}
\begin{equation}
\frac{l_{01}}{l_{11}}=\frac{\displaystyle\sqrt{2}\left(\frac{R_{11}}{R_{01}}\right)-1}{\displaystyle\sqrt{2}-\frac{R_{11}}{R_{01}}}.
\end{equation}
根据式(2.1),当$R_{11}/R_{01}\geq\sqrt{2}$时,二维模式晶体生长形态仅为\{01\}单形;当$R_{11}/R_{01}\leq\sqrt{2}/2$时,二维模式晶体生长形态仅为\{11\}单形;当$\sqrt{2}/2<R_{11}/R_{01}<\sqrt{2}$时,二维模式晶体生长形态为\{01\}与\{11\}两种单形所组成的聚形。

同理,对于\{001\}与\{111\}两种单形所组成的立方晶系晶体形态,不难证明,它取决于(001)与(111)两晶面的相对生长速率的比值。 %晶体生长形态与生长速率间的联系
\subsection{晶体生长的理想形态}
 %晶体生长的理想形态
\subsection{晶体生长的实际形态}
 %晶体生长的实际形态
\subsection{晶体几何形态与其内部结构间的联系}
\paragraph{(1)晶体几何形态的表示方式}根据晶体学有理指数定律,晶体几何形态所出现的晶面符号$(hkl)$或晶棱符号$[hvw]$是一组互质的简单整数。

按照Bravais法则,当晶体生长到最后阶段而保留下来的一些主要晶面是具有面网密度较高,而面网间距$d_{hkl}$较大的晶面。

晶体生长形态的变化,除同质多相的晶体外,不仅与晶体生长条件有关,而且也能反映出晶体结构的一些信息。

从X射线晶体结构分析结果可知,不管是高级晶系或是中、低级晶系晶体,晶格面网间距$d_{hkl}$、晶格常数($a,b,c,\alpha,\beta,\gamma$)和面网族$\{hkl\}$三者之间存在着一定的关系,例如,对于立方面心晶格晶体,面网间距$d_{hkl}$、晶格常数$a$和面网族$\{hkl\}$三者之间存在着如下关系:

当$h,k,l$全为奇数或全为偶数时
\begin{equation}
d_{hkl}=\frac{a}{\sqrt{h^2+k^2+l^2}}
\end{equation}

当$h,k,l$中有奇数也有偶数时
\begin{equation}
d_{hkl}=\frac{a}{2\sqrt{h^2+k^2+l^2}}
\end{equation}
从式(2.2)和式(2.3)中可得出,$a^2/d^2_{hkl}$值随$h,k,l$值变化是有规律性的,见表2.1。

(表2.1\quad$a^2/d^2_{hkl}$值随$h,k,l$值的变化规律)

根据上述Bravais法则,属于这种结构的晶体,出现在晶体形态中的单形顺序应为$\{111\},\{100\},\{110\},\cdots$。天然的萤石($\rm CaF_2$)和金刚石(C)等晶体,基本上是符合上述规律的。但对于中、低级晶系晶体,尤其是对于低级晶系晶体,面网间距$d_{hkl}$、晶格常数($a,b,c,\alpha,\beta,\gamma$)和面网族$\{hkl\}$三者之间的关系就复杂化了,但$d_{hkl}$值随着面网族$\{hkl\}$的减小而增大的一般倾向却仍然存在。更值得注意的是,当晶体结构中存在着螺旋轴和滑移面对称性时,则情况变得更加复杂了。这时候必须对面网间距$d_{hkl}$进行修正,否则计算结果与实际情况就会表示不符。

当晶体结构中具有螺旋轴时,只影响与其垂直的面网间距$d_{hkl}$值,而修正因子$\beta$应随螺旋轴的轴次不同而不同,对于:
\begin{align*}
&2_1,4_2,6_3\text{ 螺旋轴}, &\beta = 1/2, \\
&3_1,3_2,6_2,6_4\text{ 螺旋轴}, &\beta = 1/3,\\
&4_1,4_3\text{ 螺旋轴}, &\beta = 1/4,\\
&6_1,6_5\text{ 螺旋轴}, &\beta = 1/6.
\end{align*}

例如属于三方晶系的石英($\alpha\text{-}\rm SiO_2$)晶体,若不考虑其结构中的$3_1$或$3_2$螺旋轴时,根据Bravais法则的推论,$z\{0001\}$单形应在晶体形态中出现的比重最大,但实际上所出现的概率却很少,若考虑到垂直于$z\{0001\}$单形的$3_1$或$3_2$螺旋轴的重要作用时,而$z\{0001\}$单形应改为$z\{0003\}$单形,这样对于$z\{0001\}$单形很少出现的原因,就可以得到合理的解释。

同样,当晶体结构中存在着滑移面时,对晶体形态中所出现的单形比重次序也能发生影响,可以证明,与(001)面平行的滑移面,只能对$\{hkl\}$单形范围内的某些单形发生影响。例如$a$滑移面或$b$滑移面,仅对$h$或$k$为奇数的$\{hko\}$单形产生影响,这由于等同部分形成的面网间距$d_{hkl}$与密度都比具有普通对称面存在时少$1/2$的原因\footnote{此处费解。}。同样,不难理解,当晶体结构中具有$n$滑移面时,仅对$(h+k)$为奇数的$\{hko\}$单形产生影响,而对$d$滑移面,仅对$(h+k)$不等于4的倍数的$\{hko\}$单形产生影响。

上述所讨论的晶体形态与晶体结构间的联系只能看作是一个粗略的轮廓,它是晶体结构对称性在其形态上反映的一些信息,但在实际情况下,由于晶体生长的外界因素及其结构基元间键合作用的影响,即便是同一品种的晶体生长形态往往也会有所不同,这样就远非像Bravais法则所推论的那样简单了。

\paragraph{(2)周期键链理论(periodic bond chain, PBC)} Hartman和Perdok等在探索晶体形态与其结构的关系时,提出了周期键链理论,此理论是在晶体化学基础上建立起来的晶体形态理论,对于具有复杂结构的晶体实验观察表明,其生长形态是可以用周期键链理论来阐明的。此理论的基本假设是,在晶体生长过程中,于生长界面上形成一个键所需要的时间随着键合能的增加而减少,因而生长界面的法向生长速率随键合能的增加而增加。由于键合能的大小决定了生长界面的法向生长速率,故键合能的大小也就决定的晶体的生长形态。该理论认为晶体结构是由周期键链(PBC)所组成的,晶体生长最快的方向是化学键最强的方向,晶体生长是在没有中断的强键链存在的方向上,这里所说的强键是在晶体生长过程中形成的强键。晶体生长过程所能出现的晶面可划分为三种类型,即$F$面、$S$面、$K$面。划分面的标准为

$F$面:或称平坦面(flat faces),它包含两个或两个以上的共面的PBC($\bm{PBC}$矢量)。

$S$面:或称台阶面(stepped faces),它包含一个PBC($\bm{PBC}$矢量)。

$K$面:或称扭折面(kinked faces),它不包含PBC($\bm{PBC}$矢量)。

所设想的PBC模型,如图2.2所示。

在图2.2中,假设晶体中具有三种$\bm{PBC}$矢量,其中$\bm{A}\text{ 矢量}\parallelsum [100]$、$\bm{B}\text{ 矢量}\parallelsum [010]$、$\bm{C}\text{ 矢量}\parallelsum [001]$方向。这些$\bm{PBC}$矢量确定了六个$F$面,即$(001),(00\bar{1});(010),(0\bar{1}0);(100),(\bar{1}00)$面;三个$S$面即$(011),(101)$和$(110)$面;一个$K$面,即$(111)$面。从图2.2中还可看出,一个结构基元生长在$F$面上,只形成一个不在$F$面上的$\bm{PBC}$矢量;一个结构基元生长在$S$面上,形成的强键比$F$面上的数目多;而在$K$面上形成的强键数最多。因此,$F$面的生长速率最慢,$S$面的生长速率次之,而$K$面生长速率最快,因而$K$面是易于消失的晶面。晶体生长的最终形态多为$F$面包围,其余的为$S$面。

 %晶体几何形态与其内部结构间的联系
\subsection{环境相对晶体形态影响}
同一品种的晶体,由于晶体生长环境相的不同,往往会出现不同的晶体形态。现以自由生长体系为例来说明生长条件不同对晶体生长形态的影响。

\paragraph{(1)溶剂的影响}在过去漫长的年代里,人们对晶体形态的研究,大多花在矿物晶体和人工无机化合物晶体上,近20多年来,由于对有机非线性光学晶体的研究受到了重视,从而也开始注意到对有机晶体的生长形态的研究。现仅以{\CJKsetecglue{} 3-甲基-4-硝基吡啶-1-氧}晶体,(简称POM晶体),为例来说明溶剂对晶体形态的影响。

溶液法是研制块状有机非线性光学晶体的主要方法之一。与无机晶体水溶液生长不同之点是,有机晶体可选择多种不同有机非水溶剂。溶液中溶质与溶剂之间的相互作用对晶体生长过程有着极为重要的影响,因此,可从分子水平的晶体微观结构来研究{\CJKsetecglue{}溶质-溶剂}间的相互作用,研究晶体生长的基元化过程与晶体生长的脱溶剂化过程,进而研究晶体的生长机制与生长动力学规律,这些研究对有机晶体生长理论的发展和实际应用均有重要意义。

POM晶体时硝基吡啶类有机分子晶体,它比其他芳香硝基化合物的紫外截止波长更短,并具有较大的倍频系数,是研制非线性光学器件的候选材料。POM可溶于水、乙醇、甲苯、苯、丙酮、乙腈、环己酮、二甲基甲酰胺、二甲亚砜等溶剂中,因此,要想生长出优质大尺寸的POM晶体,优选最佳化的溶剂是一个很重要的生长条件。

溶质与溶剂间相互作用不仅影响溶液的溶解度,而且对晶体的生长形态会产生很大的影响。从环己酮、二甲基甲酰胺和二甲亚砜三种不同溶剂中生长出来的POM,其晶体形态如图2.6所示。

(图2.6)

从图2.6中可看出,POM晶体的生长形态主要是由$\{100\},\{210\},\{302\}$等单形所组成。之所以能出现不同的生长形态,可能是由于溶剂分子与某一晶面上溶质分子具有较强的选择吸附作用,难于脱溶剂化,从而降低了该晶面的生长速率,其结果便引起了晶体生长形态的变化。 %环境相对晶体形态影响

\section{晶体生长的输运过程}
晶体生长包括一系列过程,诸如晶体生长基元形成过程、晶体生长的输运过程、晶体生长界面动力学过程等,其中输运过程是一个重要的环节。从宏观的观点来看,晶体生长过程实际上是一个热量、质量和动量的输运过程。晶体生长是空间不连续与非均匀化的过程,结晶作用仅在生长界面上发生。对熔体生长来说,界面的推进是通过温度梯度进行的。要想生长出优质完整的晶体,在熔体生长体系中,必须建立合理的温度梯度分布,因为它和热量输运中的热传输机制相联系,晶体生长时所释放出的结晶潜热必须及时地从界面处输运出去,然后才能发生凝固过程。晶体从气相或溶液中生长时,生长基元首先从生长体系的其他部位输运到生长界面,然后再进行生长基元脱溶剂化、脱附、表面扩散、生长基元进入生长位置等界面反应过程。

晶体的生长输运过程主要包括热量、质量和动量输运等。这些效应均称为输运效应,而各种不同形式的输运效应是相互联系的,例如气相各部分的运动速度可以彼此不同而造成动量输运,但在气相动量运输的同时会引起质量的输运;同样,在气相中,由于温度梯度而造成的热量输运也会引起质量输运效应。

晶体生长的输运过程对其生长速率产生限制作用,并支配着生长界面的稳定性,对于生长晶体的质量有着极其重要的作用。
晶体生长的输运理论是流体力学理论发展的一个重要组成部分,多年来一直是人们所关注的研究课题,并做了不少有意义的理论与实验工作,然而有关这一方面定量的研究结果至今还不多,造成这种状况的原因是多方面的,例如对粘滞性液体动力学方程求解的困难,特别是对于一般晶体生长过程中的实际边界条件,方程式的求解基本上是不大可能的;其次,由于受某些晶体生长条件的限制(高温、高压、密封、液体不透明等),使直接观察与测量造成困难,这也是影响深入研究的原因。

\subsection{输运的类型}
 %输运的类型
\subsection{边界层理论}
 %边界层理论

\section{晶体生长界面的稳定性}

\subsection{研究界面稳定性应遵循的几个原则}
 %研究界面稳定性应遵循的几个原则
\subsection{生长界面稳定性的判据}
生长晶体时为了使界面稳定,寻求界面稳定性的判据显然是很重要的,确定界面是否稳定,可通过界面附近熔体的温度梯度、溶液中溶\footnote{原文作“熔”。}质的浓度梯度和界面效应等途径来做出判断的。

\paragraph{(1)熔体的温度梯度}晶体生长界面前沿的温度梯度,不外乎三种情况,一种是正温度梯度,即$(dT_l/dx)>0$,$x$的方向指向熔体,这样的熔体称为过热熔体。另一种情况称为负温度梯度,即$(dT_l/dx)<0$,这样的熔体称为过冷熔体。第三种情况是不常见的,即界面前沿的温度为熔体熔点温度,这时$\dfrac{dT_l}{dx}=0$。

对于过热熔体,如果平坦界面在偶然的外界因素干扰下而出现了凹凸不平,但由于离开界面的熔体温度梯度为正值,界面的凸起部位必然处于较高的温度$T_l$。由于$T_l>T_0$($T_0$为熔体的凝固点温度),在这种情况下,界面凸起部分的生长速率便会逐渐降低,从而被界面凹入部位的生长所追及,最后生长界面必然恢复到原来的光滑面的状态,因此这种类型的界面是稳定的,这样,熔体中的正温度梯度是有利于界面稳定性的因素,并可作为生长界面稳定性的判据。根据界面上能量(热)守恒原则
$$\kappa_s\frac{\partial T_s}{\partial x}=\kappa_l\frac{\partial T_l}{\partial x}+f\rho L$$
由于$\dfrac{\partial T_l}{\partial x}>0$,可用生长速率$f$的大小来作为界面稳定性的判据
\begin{equation}
f<\frac{\kappa_s}{L\rho}\frac{\partial T_s}{\partial x},
\end{equation}
式中$\rho$为晶体密度,$L$为结晶潜热。

对于过冷的熔体,熔体中的温度梯度为负值,即$(\partial T_l/\partial x)<0$,这时熔体中的温度$T_l$低于熔点温度$T_0$,在这种情况下,有利于平坦界面受外界因素干扰而产生的凸起部位的生长,这种类型的界面显然是不稳定的,因此熔体中的负温度梯度是不利于界面稳定性的因素。

当熔体中的温度等于熔点温度时,即$(\partial T_l/\partial x)=0$,整个熔体温度时均匀分布,在这种情况下,平坦界面是否稳定,那就要看平坦界面所受外界干扰大小而定,当干扰大时,平坦界面也能变为不稳定的。

\paragraph{(2)溶质浓度梯度}如前所述,当界面前沿的熔体为正温度梯度,即$(\partial T_l/\partial x)>0$时,界面稳定,但这只限于对纯熔体而言的,而完全纯的熔体实际上是不存在的。如果考虑到溶质浓度梯度,即便是熔体的温度梯度为正值,平坦界面也有可能是不稳定的。

(图2.13 界面处溶质浓度的分布)

当熔体中含有平衡分凝系数$k_0<1$的溶质时,在晶体生长过程中,这些多余的溶质会在界面上不断地汇集,而形成溶质边界层$\delta_c$。当边界层愈接近界面时,其溶质浓度愈高,如图2.13所示。由于溶质在界面处浓集,致使熔体的凝固点温度降低,这时,晶体生长体系中熔体的凝固点温度分布可用下式表示:
\begin{equation}
T_l=T_0+mc_i\left[1+\left(\frac{1-k_0}{k_0}\right)\exp(-vx/D_l)\right],
\end{equation}
式中,$T_0$为纯熔体的凝固点温度,$m$为液相线斜率,$c_i$为晶体中溶质浓度,$k_0$为溶质的平衡分凝系数,$v$为晶体生长速率,$x$为界面指向熔体的运动坐标,$D_l$为溶质的扩散系数。这时在靠近界面处的熔体温度,可能发生两种不同的温度分布情况,如图2.14所示。

(图2.14 界面处熔体中温度分布)

在图2.14中,$T_mT_L$代表熔体的凝固点温度分布,$T_mT_A$代表实际熔体具有较大的正温度梯度线,$T_mT_B$代表实际熔体具有较小的正温度梯度线,$T_m$代表熔体的凝固点,$\delta_c$代表熔质边界层厚度。

从图2.14中可看出,如果熔体具有$T_mT_B$线所代表的正温度梯度,在溶质边界层$\delta_c$内,熔体的实际温度比熔体应具有的凝固点温度低,这样在界面附近便形成了过冷区(阴影线区),在这个区域中的熔体处于过冷状态,这种过冷区不是由于熔体中的负温度梯度产生的,而是因为溶质在界面附近的熔体中浓集而引起的,故称此为组分过冷。在界面上的实际温度与熔体的凝固点温度相等,而处于平衡状态,在$\delta_c$外其余部位的熔体则处于过热状态。

从图2.14中还可看出,在界面附近的过冷区内,随着$x$的增加过冷度亦增大,在这种情况下,当界面上出现的任何干扰,都会在过冷区内迅速地长大,这样一来,原来为光滑平坦界面就会变为凸凹不平的胞状的不稳定界面。如果熔体具有$T_mT_A$线所代表的较大正温度梯度时,在$T_mT_A$线上任何一点的温度都高于熔体应有的温度,因此,界面前沿均不会出现组分过冷现象,界面仍可保持稳定。

根据实际的熔体正温度梯度($T_mT_A$线和熔体的应有的平衡温度线($T_mT_L$)在界面上相切的条件,即实际的温度梯度线同平衡温度线具有相同的斜率,便可求出熔体不产生组分过冷的临界条件。这个实际的熔体温度梯度($G_l$)为溶质浓度曲线的斜率$dc_l/dx$与相图的液相线的斜率$m$的乘积。

求解界面处一维扩散方程,即可得到溶质浓度曲线的斜率$dc_l/dx$值
$$D_l\frac{\partial c_l}{\partial x}=(c_{l(0)}-c_s)v,$$
\begin{equation}
\frac{\partial c_l}{\partial x}=[c_{l(0)}-c_s]\cdot\frac{v}{D_L}.
\end{equation}
但是
$$c_s=k^*c_{l(0)}=k_ec_l,$$
所以
$$\frac{\partial c_l}{\partial x}=c_l(l-k^*)v\frac{k_e}{k^*D_l},$$
式中,$c_l$为熔体整体浓度,$k^*$为界面溶质分凝系数,$k_e$为有效分凝系数,$v$为晶体生长速率。因此,不产生熔体组分过冷的临界条件为
\begin{equation}
\frac{G_l}{v}\geq\frac{mc_l(1-k^*)k_e}{k^*D_l}.
\end{equation}
从式(2.41)中可看出,为了克服组分过冷,至少应当注意三种因素的影响,即荣体重的温度梯度($G_l$)、晶体的生长速率($v$)和溶质浓度($c_l$)的影响,以便有利于保持界面的稳定性。

\paragraph{(3)界面能效应}当晶体生长时,如果界面的位移面积发生变化,则相应的界面能也要发生变化。当界面为平坦面时,在偶然因素干扰下会产生凸起,这样便增加了界面的面积,从而使界面的总界面能增大,由于界面能的增加就提高了体系的自由能,但体系的自由能总是向减少的方向进行,而使界面面积趋于缩小,这将使平坦面所产生的凸起趋于消失。因此,界面能对界面稳定性是有贡献的。

当界面两侧的固相压强$P_s$与液相压强$P_l$不等时,如$P_s\neq P_l$时便存在着液面的压力差
\begin{equation}
\delta_p=P_s-P_l.
\end{equation}
如果$P_s>P_l$,$\delta_p$为正值,这样界面的曲率半径$r$的中心在晶体内,界面凸向熔体。当界面为平坦面时,这时$P_s=P_l$,$\delta_p=0$,这样便没有附加的界面压力,较为稳定。如果$P_s<P_l$,$\delta_p$为负值,那么界面的曲率半径$r$的中心在熔体内,而界面凸向晶体,这是一般所不希望的。出现界面压差$\delta_p$的原因,主要来自于界面为曲面时的平衡参量与界面为平坦面时的平衡参量之间的差异。界面效应如图2.15所示。

(图2.15 界面效应) %生长界面稳定性的判据
\subsection{界面稳定性动力学理论}
 %界面稳定性动力学理论

\clearpage
\section{晶体生长界面结构理论模型}

\subsection{完整光滑面理论模型}
 %完整光滑面理论模型
\subsection{非完整光滑面理论模型}
 %非完整光滑面理论模型
\subsection{粗糙界面理论模型}
 %粗糙界面理论模型
\subsection{扩散界面理论模型}
 %扩散界面理论模型


\section{晶体生长界面动力学}

\subsection{完整光滑面的生长}
 %完整光滑面的生长
\subsection{非完整光滑面的生长}
 %非完整光滑面的生长
\subsection{粗糙界面的生长}
 %粗糙界面的生长
\subsection{扩散界面的生长}
 %扩散界面的生长
\subsection{BCF体扩散理论和GGC体─表面耦合扩散理论}
 %BCF体扩散理论和GGC体-表面耦合扩散理论
\subsection{高聚物晶体生长}
 %高聚物晶体生长

\section{晶体生长动力学研究实验方法}

\subsection{等组分方法}
 %等组分方法
\subsection{原位实时观察法}
 %原位实时观察法
\subsection{显微照相法}
 %显微照相法
\subsection{光散射法}
 %光散射法
\subsection{计算机模拟法}
 %计算机模拟法

\clearpage
\section{空间晶体生长的微重力效应}

\subsection{微重力概念}
 %微重力概念
\subsection{空间微重力晶体生长实验}
 %空间微重力晶体生长实验
\subsection{微重力状态下的输运过程}
重力对流体的输运性质有重要影响。这种影响是宏观的,主要与重力驱动的对流和由阿基米德力引起的上浮和下沉有关,因而流体的所有输运性质都与微重力这个概念密切相关。

晶体生长过程是固体结晶基元和流体结晶基元在特定条件下的相互作用和能量再分配过程。结晶过程总是离不开固体成分和流体成分。因固体内部结合力远大于1个$g_0$的重力,所以结晶过程仅需考虑流体成分受重力的影响,在溶液中流体的粘着力和表面张力与$g_0$的重力同数量级,液体的常规性质是内分子力和重力相互作用的结果。假如重力消失,流体的行为将受内分子力支配。因而,微重力条件将影响晶体的生长过程。结晶体系中能量的再分配离不开质量的输运和热量的传递,质量的输运通过对流和扩散途径进行,而热量传递则通过对流传导和辐射途径进行。在微重力条件下,扩散将支配质量输运过程。晶体生长的输运问题将变成Stefan问题。在数学物理方程的处理上就变成求解自由边界和运动边界的扩散微分方程问题。基于上述原因,为了认识微重力条件下晶体生长的动力学特征,研究扩散引起的质量输运的理论和实验现象就十分必要了。

\paragraph{(1)扩散理论}扩散是气体、液体和固体中的普遍现象。扩散系数的典型值对气体是$\rm 1\ cm^2/s$,对液体是$10^{-5}\ \rm cm^2/s$,对固体通常则远低于$10^{-8}\ \rm cm^2/s$。解释扩散现象的基本定律是Fick第一定律和第二定律,如2.2.3节所述。流体中的扩散理论模型可归纳如下:

\begin{enumerate}[(i)]\itemsep -0.5ex
\item 准晶模型。假设流体中的原子是按照类似于在固体里的排列方式,原子或分子的扩散必须克服激活势垒。扩散系数和温度的函数关系,服从(Arrhenius)方程(2.125)
\begin{equation}
D=D_0\exp(-\frac{Q}{kT}),
\end{equation}
式中$D_0$是指数前因子,$Q$是激活能,$k$是玻尔兹曼常数,$T$为绝对温度。

\item 自由体积模型。假设原子或分子扩散到与其相邻的自由体积内,自由体积具有临界体积$V_c$,扩散系数$D$和温度遵循下列方程:
\begin{equation}
D=AT\exp(-\frac{BV_c}{V_f}),
\end{equation}
式中$A,B$是与材料物性有关的参数,$V_f$是液体的总体积。

\item 涨落理论。假设原子或原子集团由于热振动(布朗运动)而进入周围自由体积内,而且自由体积没有临界尺寸。扩散系数和温度满足二次函数关系
\begin{equation}
D=AT^2,
\end{equation}
式中$A$是与原子相互作用势能有关的常数。

\item 能斯脱-爱因斯坦-斯托克斯(Nernst-Einstein-Stokes)关系。假设扩散系数可以表示为驱动力和摩擦力之比,即
\begin{equation}
D=\frac{kT}{6\pi\eta r},
\end{equation}
式中$\eta$为粘滞系数,$r$是扩散原子的半径。
\end{enumerate}

在地面上,由于重力驱动对流的存在使得地面实验结果不准确,甚至还不能决定哪些模型可正确地描述其与温度的依赖关系。在空间环境中,重力驱动的自然对流大大减弱或得到消除,因此可完成一些精度足够高的实验,并可精确地测定由扩散控制的质量输运性质,进而导致关于流体中扩散和输运的新的理论概念,也会促进地面晶体生长技术的变革和生长理论的完善。科学家还语言微重力条件下将会出现新晶体和新理论。

\paragraph{(2)空间实验结果}
\begin{enumerate}[(i)]\itemsep -0.5ex
\item 自扩散实验。1983年11月在空间实验室-1中进行了$\rm Sn$的自扩散实验。用16个样品,作了不同温度下的8次实验。每次用直径分别为$\rm 1\ mm$和$\rm 3\ mm$的两个样品。每个样品长$\rm 55\ mm$,在其一端固定$\rm 4\ mm$厚的稳态$\rm Sn$同位素圆片。稳态的同位素$\rm^{124}Sn$和$^{112}Sn$用来检测自扩散系数和同位素效应。表2.10列出了$\rm Sn$的自扩散实验结果。

(表2.10)

在表2.10中$E$为同位素效应。$D_{112}$和$D_{124}$分别为两种Sn同位素的扩散系数。从空间实验获得的扩散系数比地面值低20-40\%。这表明在地面上进行$D$的测量时,对流有明显影响。另一个重要结果是扩散系数和温度的关系为二次函数,即

(图2.51)

\begin{equation}
D({\rm Sn})=AT^2,
\end{equation}
式中$A=0.745\times 10^{-16}\ \rm cm^2/k^2s$,即在$\pm2\%$的精度内,扩散系数和温度是平方关系。这个结果支持了涨落理论。

\item 互扩散实验。空间条件下的玻璃熔体实验首先是在探空火箭上进行的,$\rm Na_2O\cdot3SiO_2$和$\rm Rb_2O\cdot3SiO_2$熔体在$1200$℃温度下持续$\rm 200\ s$。后来在空间实验室(LS-1)上重做了实验。实验温度$\rm 1280$℃,持续时间$\rm 1730\ s$。图2.51示出微重力条件下的实验结果。图2.52示出相同条件的地面实验结果。这两个对比实验表明:在地面上由于有对流的存在,Na和Rb的浓度分布曲线不规则,用它几乎不可能测定有意义的扩散系数。而在微重下,理论数据和实验数据符合得很好。空间实验数据分析还表明:当Na和Rb的浓度相同时,扩散系数最大。

(图2.52)

\item GeSe和GeTe的质量输运速率。在天空实验室(Skylab-3)上已研究了IV-VI族半导体GeSe和GeTe的化学输运反应。用$\rm GeI_4$作为输运载体。密封石英管的尺寸为$\Phi\rm13\times150\ mm$。在实验过程中,先升温$\rm3\ h$,再在生长温度下恒温$\rm33\ h$,然后降温到室温,降温过程历时$\rm12\ h$。表2.11列出了空间和地基实验数据。

(表2.11)

\noindent 表2.11中,$A,B$和$C$是样品编号。试验结果表明,$\rm\mu g$下的质量输运速率依赖于材料和载体气压,其数值可以小于、大于或等于$1\rm\ g$条件下的实验值。
\end{enumerate} %微重力状态下的输运过程
\subsection{微重力环境下的晶体生长形态}
 %微重力环境下的晶体生长形态
\subsection{微重力对晶体组分、结构和完整性的影响}
 %微重力对晶体组分、结构和完整性的影响

