\setcounter{chapter}{2}
\chapter{溶液法生长晶体}
\authors{高樟寿\quad 蒋民华\quad 王希敏}
从溶液中生长晶体的历史最久,应用也很广泛。这种方法的基本原理是将原料(溶质)溶解在溶剂中,采取适当的措施造成溶液的过饱和状态,使晶体在其中生长。溶液法具有以下优点:
\begin{enumerate}[(1)]
\item 晶体可在远低于其熔点的温度下生长。有许多晶体不到熔点就分解或发生不希望有的晶型转变,有的在熔化时有很高的蒸汽压,溶液使这些晶体可以在较低的温度下生长,从而避免了上述问题。此外,在低温下使晶体生长的热源和生长容器也较易选择。
\item 降低粘度。有些晶体在熔化状态时粘度很大,冷却时不能形成晶体而成为玻璃体,溶液法采用低粘度的溶剂则可避免这一问题。
\item 容易长成大块的、均匀性良好的晶体,并且有较完整的外形。
\item 在多数情况下,可直接观察晶体生长过程,便于对晶体生长动力学的研究。
\end{enumerate}

溶液法的缺点是组分多,影响晶体生长因素比较复杂,生长速度慢,周期长(一般需要数十天乃至一年以上)。另外,溶液法生产晶体对控温精度要求较高。在一定的温度($T$)下,温度波动($\Delta T$)对晶体生长的影响取决于$\Delta T/T$,若维持$\Delta T/T$数值不变,则在低温下$\Delta T$应当小。经验表明,为培养高质量的晶体,温度波动一般不宜超过百分之几,甚至是千分之几度。

溶液生长的范畴包括水溶液、有机溶剂和其他无机溶剂的溶液、熔盐(高温溶液)以及水热条件下的溶液等。本章只着重讨论从水溶液中生长晶体的问题。

\setcounter{section}{0}
\section{溶液和溶解度}

\subsection{溶液的概念}
 %溶液的概念
\subsection{溶解度和溶解度曲线}
溶解度是考察溶液中生长晶体的最基本参数。溶解度可用在一定条件(温度、压力)下饱和溶液的浓度来表示。溶质在溶液中的浓度(溶液成分)有以下几种表示方式:

\begin{enumerate}[(1)]
\item 体积摩尔浓度(mol):一升溶液中所含溶质的摩尔数。
\item 重量摩尔浓度(mol):1000g溶剂中所含溶质的摩尔数。
\item 摩尔分数($x$):溶质摩尔数对溶液总摩尔数之比。
\item 重量百分数:100g(或是1000g)溶液中所含溶质的克数。
\item 重量比:100g(或是1000g)溶剂中所含溶质的克数。
\end{enumerate}

不同的浓度表示方式适用于不同的场合。在实验室中使用(1)表示方式是很方便的,但由于和溶液体积有关,易受温度影响(某一给定的摩尔随温度升高而减小),因此在溶解度数据中,经常使用其他浓度表示法,最常用的的是重量比和摩尔分数,后者特别适于多组分混合物的成分。

在数种质组成的溶液中,某一组分的摩尔分数可表示为
\begin{equation}
x_1=\frac{m_1/M_1}{m_1/M_1+m_2/M_2+m_3/M_3+\cdots},
\end{equation}
$m$是组分的重量,$M$是其分子量。任何混合物中所有成分的摩尔分数总和等于1。

各种浓度表示法可以互相换算,对于摩尔与其他浓度表示法换算时还需知道溶液的密度。在水溶液中,当溶质含有水合物和无水合物两种形式时,其重量百分数和重量比的换算公式如下:
\begin{equation}
c_1=\frac{100c_2}{100-c_2}=\frac{100c_3}{100R-c_3}=\frac{100c_4}{100R+c_4(R-1)},
\end{equation}
其中$c_1$为100g水中所含无水物的克数,$c_2$为100g溶液中所含无水物的克数,$c_3$为100g溶液中所含水合物的克数,$c_4$为100g水中所含水合物的克数;$R$=水合物分子量/无水合物分子量。

在我们所讨论的溶液体系中,压力对溶解度影响是很小的,但温度的影响却十分显著。这种温度-浓度关系可用溶解度曲线表示。图3.3示出一些常用水溶性晶体的溶解度曲线。从图3.3中可以看出,不同物质在水中的溶解度是有明显差别的。

%TODO: 图3.3
(图3.3)
% 图3.3一些水溶性晶体的溶解度曲线
%1.酒石酸钾钠(KNT); 2 .酒石酸钾(DKT); 3,酒石酸乙二胺(EDT);
%4.磷酸二氢铵(ADP);5.硫酸甘氨酸(TGS); 6.碘酸锂((L);7.磷酸二氢钾(KDP); 8.硫酸锂(LSH)。

有的晶体溶解度很大,如DKT,但有的则较小,如LSH,而大多数晶体的溶解度均随温度升高而增大(溶解度温度系数为正值)。有的晶体溶解度温度系数很大(如KNT),但也有少数晶体(如LSH,Li等),溶解度的温度系数很小,而且是负值。有的晶体溶解度曲线出现拐点(如EDT),这种物质有两种不同的溶解度晶相,即EDT无水物和水合物。图3.3中所示的溶解度曲线实际上是由EDT和$\rm EDT\cdot H_2O$两条溶解度曲线组成的,这两条曲线在拐点(40.6℃)相交。该点的温度称为相平衡转变温度。

溶解度曲线是选择从溶液中生长晶体的方法和生长温度区间的重要依据。如对于溶解度及其温度系数都很大的物质,采用降温法比较理想,而对于溶解度较大,而温度系数却较小的物质则宜采用蒸发法,对于具有不同晶相的物质则需选择对所需要的那种晶相是稳定的合适的生长温度区间。

温度对溶解度的影响可用下方程式表示:
\begin{equation}
\frac{d\ln{x}}{dT}=\frac{-\Delta H}{RT^2},
\end{equation}
式中$x$为溶质的摩尔分数,$\Delta H$为固体摩尔溶解热,$T$为绝对温度,$R$为气体常量。在理想情况下,式(3.3)可化为
\begin{equation}
\log x=\frac{-\Delta H}{2.303R}\left( \frac{1}{T}-\frac{1}{T_0}\right) =\frac{-\Delta H(T_0-T)}{4.579T_0T},
\end{equation}
$T_0$为晶体的熔点。从式(3.4)中可看出:
\begin{enumerate}[(1)]
\item 大多数晶体的溶解过程是吸热过程,为正,温度升高,溶解度增大。如若是放热过程,则相反。
\item 在一定温度下,高熔点晶体的溶解度小于低熔点的溶解度。
\end{enumerate}

式(3.4)也可写成
\begin{equation}
\log x = -\frac{a}{T} + b,
\end{equation}
式中$a$和$b$都是常数。

温度对溶解度的影响也可表示为
\begin{equation}
c=A+Bt+Ct^2+\cdots,
\end{equation}
式中$c$为一定量溶剂中溶质的重量,$A,B,C$是和溶液体系(溶质-溶剂)有关的常数。式(3.5)和式(3.6)是表示温度对溶解度影响的最常用的表达式。

%TODO: 图3.4
(图3.4)

根据式(3.5)作图得出的溶解度曲线如图3.4(b)所示。图中的横坐标为溶质摩尔分数的对数坐标。右边纵坐标是(因为在273-373K范围内数值很小)的线性坐标。另外,也可以用专门的对数-倒数坐标图纸,将温度(℃)直接在图上标出。图3.4(b)中左边示出的纵坐标就是倒数标尺上直接标出的温度。将图3.4(a),(b)加以比较就可发现这种作图法有如下的一些优越性:
\begin{enumerate}[(1)]
\item 线性好。在图3.4(a)中,在0-100℃范围内,CuSO4是平滑曲线,$\rm Na_2SO_4$和$\rm Na_2CrO_4$则分别是两条曲线在拐点(转变点)相交,而在图3.4(b)中,上述三种盐的溶解度曲线都是直线,但对一些溶解度很大的物质,或在很窄的温度区间内存在着数种水合物的物质,其溶解度仍是曲线。醋酸钠在40—50℃ 范围内直线发生弯曲,可能属于后一种情况。
\item 记录范围宽。对一些溶解度很大的盐类(如铬酸钠和醋酸钠的溶解度曲线)来说,都能像溶解度较小的盐类一样,在0-100 ℃范围内可出,这是通常的作图法无法做到的。
\item 较清楚地显示出转变点。例如通过两直线交点找$\rm Na_2SO_4$的转变点(32.4℃)要比图3.4(a)中所示延长两条曲线找转变点来得容易和准确。在图3.4(b)中,$\rm CuSO_4$的两条直线大约在67℃相交,表明硫酸铜在该处有不同的晶相转变,而在图3.4(a)中却找不出这一转变点。
\end{enumerate}

基于上述优点,利用$\log{x}-1/T$图可以用内插法和外推法求出溶解度的数值,并估计出转变温度(如果存在相平衡转变)。因此,式(3.5)在实践中很有用。 %溶解度和溶解度曲线
\subsection{饱和与过饱和}
 %饱和与过饱和
\subsection{溶液饱和温度、溶解度和过饱和度的测定}
溶液到达饱和状态时的温度,即溶质固体和溶液达成平衡的温度称为溶液的饱和温度。准确地测定溶液的饱和温度是搞好下种操作的前提,也是测定溶解度和过饱和度的基础,所以饱和温度的测定是从溶液中培养晶体的一项基本功。常用的测定饱和温度的方法有以下几种:

\paragraph{(1) 平衡法}
在接近饱和的溶液中,放人一些溶质固体,在一定温度下不断搅拌,直到溶液中尚余少量固体不再溶解为止,此时溶液的温度即可看成是溶液的饱和温度。这个方法虽然简便,但要达到真正平衡所需的要达到真正平衡所需的时间较长(约数小时至数天,视溶液的粘度和搅拌强度而定),而且精确度也较低,约0.5—1℃。

\paragraph{(2) 浓度涡流法}
\begin{figure}[thb]
 \centering
 \includegraphics[width=0.4\textwidth]{fig/cp03/img3.8.jpg}
 \caption{浓度涡流。}
\end{figure}
%TODO:Fig.3.8
用尼龙线将一小块晶体悬在其接近饱和温度的溶液中,仔细观察晶体及其附近的液流情况。如果溶液是不饱和的,则晶体稜角变得圆滑。靠近晶体表面的溶液,由于晶体的溶解,其浓度比周围溶液浓度大,因而变得较重而向下运动,形成一股向下的液流, 并把这股液流称为溶解涡流。如果溶液是过饱和的,则晶体呈现生长 现象,晶面变光滑,稜角“发毛”变白。晶体附近的溶液由于溶质在晶体上析出,密度变小,因而形成一股向上运动的液流,称为生长涡流(图3.8)。涡流是溶液中浓差造成的对流运动。距饱和温度越远,涡流越明显;离饱和温度愈近,涡流就愈微弱;在饱和温度下,涡流完全消失。因此,可以通过观察涡流的变化来确定饱和温度。在测定时,可从不饱和状态开始,逐渐降低温度,观察晶体附近的浓差涡流从溶解涡流减弱到生长涡流出现的过程,找出涡流消失时的温度,即为溶液的饱和温度。为了提高准确度,可反复数次通过饱和点进行测定。这个方法的精确度约为0.1—0.5 ℃(与观察者熟练程有关)。使用该法时,要防止溶液分层,测定前溶液应充分搅拌,测定时只让溶液发生自然对流。

\paragraph{(3) 光学效应法}
溶液接近饱和温度时,涡流十分微弱,凭肉眼要把温度测定得很精确是困难的,光学效应法可以克服这一缺点。

当晶体处于与它不相平衡的母液中时,紧贴晶体附近有一薄层溶液,晶体溶解和生长时,溶质的扩散输出和输入都通过这一薄层进行,该薄层溶液称为扩散层或结晶区。扩散层存在浓度梯度,当溶液接近饱和温度时,扩散层的浓度梯度趋于消失。溶液到达饱和温度时,扩散层消失。扩散层是浓度不均匀的区域,因此在光学上也是不均匀的,光线通过这一区域时会因折射率梯度方向不同而发生不同的偏折,光学效应法的原理就基于此。常用的光学效应法有下两种类型。

\subparagraph{(i) 纹影法}
\begin{figure}[htb]
 \centering
 \includegraphics[width=0.8\textwidth]{fig/cp03/img3.9.jpg}
 \caption{测量溶液中不均匀区域的装置。}图中:$S$为光源;$O_1,O_2,O_3$为长焦距消色差透镜;$R$为可调的孔径光阑;$K$为透光面平行的测量槽;$C$为晶体;
 
 $M$为电磁搅拌器;$E$为屏幕。
\end{figure}
%Fig.3.9
其装置如图3.9所示。这种装置适用于观察透明介质中的局部光学不均匀性。光源$S$通过$O_1$,聚焦至可调孔径光栏$R_1$即$O_2$的焦点上,光线透过$O_2$即变成平行光照射到待测的不均匀透明介质上,经$O_3$成像于焦点$S'$,在该处放置一锐边遮光板或狭缝$R_2$,挡住了光源的像$S'$,因此在白色的屏幕上出现黑色的背景,但在$SS'$之间存在光学不均匀区域$A$,则光线发生偏折,从$R_2$的锐边旁通过在屏幕上给出其像$A'$。偏折方向与不均匀区域的折射率梯度方向有关。这种方法能发现折射率差别很小的不均匀区域。在光路中的待测生长池$K$中,晶体附近的不均匀扩散层可以在屏幕上清楚地显示出来。在不饱和的溶液中,扩散层的像出现在无遮光板一侧的晶体附近,在过饱和溶液中则恰好相反,如图3.9所示。当溶液到达饱和状态时,扩散层消失。在测定时,溶液可以不停地搅拌,生长池的温度可用电热吹风机调整。这种方法的精确度可达0.05℃,甚至更高。值得注意的是,光源应用单色光,用He-Ne激光更好,否则,$O_2,O_3$应用消色差透镜。

\subparagraph{(ii) 狭缝光源法}
该法是使一狭缝光源和置于待测溶液中晶体的一个晶面斜交,随着溶液状态的变化,在狭缝和晶面的交界处会出现不同的偏折现象,这也是由于存在扩散层这一光学上不均匀的区域所引起的。

\begin{figure}[htb]
 \centering
 \includegraphics[width=0.6\textwidth]{fig/cp03/img3.10.jpg}
 \caption{狭缝光源在晶面交界处的光学效应。} 
\end{figure}
%Fig.3.10

当溶液不饱和时,明亮的狭缝在晶面交界处弯曲成钝角,当溶液为过饱和时,则向相反方弯曲成锐角。溶液愈偏离饱和状态,弯曲现象就越明显。随着溶液逐淅接近饱和,弯曲部分逐渐缩短,当溶液达到饱和时,狭缝光在晶面交界处不发生弯曲(图3.10)。根据这种现象所测定的饱和温度的精确度也可达到0.05℃。 使用这一方法时,可将盛有待测溶液的容器放在恒温槽中进行测量,也可通过特定装置将溶液从大育器中抽出,流过调温装置和测量装置,再泵浦回育晶器,这样可以迅速而准确地测定育晶器中溶液的饱和温度,在生产和实验室中使用这种方法是很方便的。

利用上述测定饱和温度的方法,特别是光学效应法,也可以进行溶解度的测定,并绘制出溶解度曲线。溶解度测定虽较简单,但需要有高度的实验技巧和精确度。

用平衡法测溶解度时,在到达平衡后应恒温静置使细小分散的固体颗粒沉降,仔细抽取一定量(不是一定体积)的溶液样品进行分析,确定溶液成分。溶液分析通常用重量法和容量法,有时也使用其他方法(如比色法以及密度、折射率、电导等方法)。溶液成分也可以预先准确配制(即称取一定量的溶剂和溶质),然后再测出溶液的饱和温度,确定其溶解度。

在确定溶液成分的同时,也有必要对与之相平衡的固相进行分析鉴定。有时在很小的温度区间内与饱和溶液相平衡的稳定相是可以改变的,转别是水合物体系。例如在0-100 ℃区间内测定$\rm Na_2CO_3$在水中的溶解度时发现,在32℃以下,其稳定相是$\rm Na_2CO_3\cdot 10H_2O$,而$\rm Na_2CO_3\cdot 7H_2O$在32.0℃到35.4℃之间稳定,$\rm Na_2CO_3\cdot H_2O$在35.4℃以上稳定。因此,必须将不同温度下的平衡固相晶体取出,在该温度下仔细于燥,并确定其成分。对于成分相同但结构不一样的不同晶相(多形体),还必须利用物理鉴定的方法来确定与溶液相平衡的物相。例如从高浓度重水($\rm D/(D+H)=99.8\%$)溶液中,生长磷酸二氘钾(DKDP)晶体时,DKDP会出现四方和单斜两种晶相。在21℃以下时,四方相是与溶液平衡的稳定相,在21℃以上时,单斜相是稳定相。这两相的溶解度十分接近(特别是在转变点附近),因此用通常测定溶解度的方法很难将它们区分开来。在这种情况下,光学效应法就显示出其优越性。图3.11示出的四方和单斜两相溶解度曲线就是用图3.9的装置测定出来的。对每一相不仅测定了在其相稳定区内(即稳定相)的溶解度曲线,还成功地测出了在另一相稳定区内(即亚稳相,参阅3.2.3节)的溶解度曲线,这样对四方相和单斜相都测出了一条完整的溶解度曲线。这两条曲线的交点即为DKDP溶解度曲线上的拐点,即四方相和单斜相的相平衡转变点(在图3.11中转变点为21 ℃)。
\begin{figure}[htb]
 \centering
 \includegraphics[width=0.6\textwidth]{fig/cp03/img3.11.jpg}
 \caption{DKDP的两相溶解度曲线。}
\end{figure}
%Fig.3.11
必须指出的是,在测定溶解度曲线的工作中,提高控温和测温的精度是十分重要的。例如在测定氯化钠在30℃水中的溶解度时,实验温度波动±0.5℃,引起溶解度测量误差约0.1\%,但在测定硫酸钠在30℃水中的溶解度时,同样的温度波动则能造成5\%的误差。所以在测定工作中,恒温器所用的温度计必须用标准温度计进行校正。

过饱和度是影响晶体生长速度和质量的重要因素,过饱和度的确定对晶体生长研究工作和培养单晶的实践都有重要意义。如果在给定温度下,溶液的浓度可以测量出来,而且相应的平衡饱和浓度是已知的,那么就不难根据式(3.7)---(3.9)来计算溶液的过饱和度。上述已提到,溶液浓度可以进行直接的分析,也可通过测量体系中某些对浓度变化敏感的性质(如密度、粘度、折射率和电导率等)来间接确定。在实验室的条件下,这些性质都可以测量得很精确。但在晶体培养过程中,要求在不破坏液体稳定性的条件下,能在育晶器中进行连续的测量。如果在生长过程中温度是变化的,还需知道被测量性质对温度的依赖关系,这样,问题就变得复杂了,因此直接应用还是较少。在上述对浓度敏感的性质中,密度和折射率对温度较不敏感。

溶液的过饱和度也可以用光学效应法在生长过程中直接测定,所得的过饱和度是以过冷度$\Delta t$表示的。这种方法的优点就是不需要知道溶液的准确成分和溶解度曲线,也不需改变育晶器的温度,在晶体生长过程中,任何时候都可测得溶液的真实过饱和度。遗憾的是,在晶体的生长过程中,抽取溶液进行循环流动,有破坏溶液稳定性,影响晶体生长的危险。但在三槽流动法中,对过热槽中的饱和度用上述方法进行测量则是十分安全和方便的。

各种过饱和度的测量方法,虽取得了一定的效果,但都存在着不少问题,有待于进一步研究解决,并希望能创造更简便、灵敏、可靠的能连续进行测量的新方法。 %溶液饱和温度、溶解度和过饱和度的测定
\subsection{溶剂的选择和水溶液的结构}
 %溶剂的选择和水溶液的结构
\setcounter{section}{1}
\section{溶液中晶体生长的平衡}

\subsection{平衡和结晶过程的驱动力}
晶体生长是一个不平衡的过程。晶体在平衡状态时既不溶解也不生长,但要研究晶体生长过程,必须掌握有关该过程中平衡状态的知识。

溶液中生长晶体最重要的参数(平衡特征)是溶解度。溶解度曲线给出溶液的饱和浓度,即与固相处于平衡状态的溶液浓度。

可以把晶体生长看成是多相化学反应。当固体物质A在溶剂中溶解并达到饱和时,可用下述化学平衡方程式描述
\begin{equation}
\mathrm{A_{\text{固}} \rightleftharpoons A_{\text{溶解}}}
\end{equation}
\begin{equation}
K=\frac{[a]_e}{[a]_e(s)}
\end{equation}
其中$[a]_e$为饱和溶液中的平衡活度;$[a]_e(s)$为固相中的平衡活度,$K$为平衡常数。通常选取标准状态,使固体物质的活度等于1,此时,浓度单位最好用摩尔分数,这样上述方程就可在多组分的体系中应用。

组分活度$a_i$与其摩尔分数$x_i$的关系为
\begin{equation}
a_i = \gamma_i x_i
\end{equation}
$\gamma_i$是某组分$i$的活度系数,若$\gamma_i=1$,则该组分服从Raoult定律。如其活度系数不等于1,但仍是常数,则该组分服从Henry定律\footnote{Raoult定律:{$p=p_0x$}。{$p$}为溶液蒸汽压,$p_0$为纯溶剂蒸汽压,$x$为物质摩尔分数;Henry定律:$p=kx$,$p$为物质在溶液上的蒸汽压,$x$为物质摩尔分数,$k$为常数。}。

在理想溶液中,溶剂和溶质都服从Raoult定律,溶液的焓和体积分别等于溶剂和溶质的焓和体积之和。这意味着在溶解时,混合热和体积变化都等于零,但溶液的熵并不等于其组分熵的总和,因为熵是随无序性增加而增加的。在非理想的稀溶液中,溶质服从Henry定律,而溶剂则服从Raoult定律。

在3.1.1节中已提到,通常把溶液中含量较大的、即摩尔分数接近1的组分看作溶剂。浓溶液(溶剂和溶质含量差不多)并不服从Raoult定律和Henry定律,活度系数和浓度有关,需根据实验数据来估计。生长晶体的溶液大多是浓溶液,因此要考虑溶质质点间的相互作用,并在此基础上进行一些计算。

溶解度与温度的依赖关系可用Van't Hoff方程表示
\begin{equation}
\frac{\mathrm{d}\ln k}{\mathrm{d}T} = \frac{-\Delta H}{RT^2}.
\end{equation}
对溶液,上式变为
\begin{equation}
\frac{\mathrm{d}\ln [a]_e}{\mathrm{d}T} = -\frac{\overline{\Delta H}}{RT^2}.
\end{equation}
$\overline{\Delta H}$为溶质A在溶剂B中的偏摩尔焓的变化量,即表示将1摩尔溶质转移到体积足够大的溶液中时(此时溶液浓度不明显偏离饱和浓度)焓的变化、即溶解热(见3.1.6节)。在服从Raoult定律的溶液中,可用$x_A$来代替组分A的活度。

在溶液里生长晶体的过程中,自由能的变化为
\begin{equation}
\Delta G=\Delta G^0+RT\ln Q,
\end{equation}
其中
$$\Delta G^0=-RT\ln k,$$
$$Q=[a]^{-1},$$
$\Delta G^0$是生长过程中自由能的变化,$k$是式(3.19)的倒数,$[a]$是组分A在过饱和溶液中的实际活度。代入式(2.23)可得
\begin{equation}
\Delta G=RT\ln\frac{[a]_e}{[a]}
\end{equation}
其中$[a]_e$是组分A在饱和溶液中的平衡活度。若活度系数$\gamma=1$或是在一定范围内,$\gamma$和浓度无关,则在式(3.22)和式(3.24)中活度可用摩尔分数或浓度代替。因此式(3.24)中的$[a]_e/[a]$可用$c^*/c$代替。$c$和$c^*$分别是溶液在一定温度下的实际浓度和饱和浓度。根据式(3.8),$c/c^*$又是衡量过饱和度大小的过饱和比$s$,所以
\begin{equation}
\Delta G=RT\ln c^*/c = -RT\ln s
\end{equation}
从式(3.25)中可以看出,在过饱和溶液中($s>1$),生长晶体的过程是自由能降低的过程。这一过程是自动进行的。过饱和是结晶过程的驱动力。$s$愈大,自由能降低也愈多。晶体生长的驱动力也愈大。
 %平衡和结晶过程的驱动力
\subsection{分配系数}
溶液生长中平衡的一个重要方面是除溶质和溶剂外的第三个组分(包括杂质、掺质和同位素置换等)在固相中的溶解度问题,也就是该组分在固相和与之相平衡的液相之间的分配问题。

如果第三个组分(以下简称物质)在溶液和固相之间达成平衡时,有
\begin{equation}
\text{物质在液态溶液中的活度}(a_l) \rightleftharpoons \text{物质在固态溶液中的活度}(a_s),
\end{equation}
则
\begin{equation}
k_0=a_s(e)/a_l(e),
\end{equation}
$k_0$称为平衡分配系数。对于接近理想溶液的稀溶液来说,有
\begin{equation}
k_0=c_s(e)/c_l(e),
\end{equation}
$c_s(e)$和$c_l(e)$分别是物质在固相和溶液中的平衡浓度,即在液相和固相分界处的浓度。显然,要达到平衡时的分配系数,就要求晶体在溶液中既不溶解也不生长,或是生长速度很慢以至接近于平衡状态。另外,物质在两相中的浓度也不宜太大,以便用浓度来代替活度。

晶体在溶液中生长时,靠近晶体表面存在着扩散层(结晶区)。物质在扩散层中的浓度和在整个溶液中的浓度是有差别的。这一差别的程度取决于分配系数$k_0$的大小以及晶体生长的速度(对于给定物质,主要取决于后者)。在使用式(3.28)时,$c_l(e)$应该是物质在扩散层中的浓度,实际上这一浓度是无法测量的,因此引入有效分配系数$k_\text{eff}$的概念,定义如下
\begin{equation} k_\text{eff} = \frac{c_s}{c_l}, \end{equation}
$c_s$和$c_l$分别是物质在固相和溶液中(远离生长界面)的实际浓度。显然,$k_\text{eff}$和晶体生长速度有关,而后者又取决于过饱和度和扩散层的厚度。当溶液中的浓度接近等于晶体界面附近溶液浓度时,即$c_l \approx c_l(e)$,则$k_\text{eff} \approx k_e$(因为在大多数情况下,$c_s \approx c_s(e)$)。

当单晶在溶液中生长时,由于晶体的各向异性,$k_\text{eff}$也和结晶方向有关。例如Fe$^{3+}$在KDP晶体的柱面和锥面上的分配系数并不一样,这样就给确定$c_s$和分配系数带来很多的困难。

总之,有效分配系数和晶体生长速度溶液的搅拌情况(影响扩散层厚度)以及结晶方向都有关系,如果测得的分配系数和上述几方面都关系不大时,我们就可认为体系接近于平衡状态,即$k_\text{eff} = k_e$。

有效分配系数还和物质的浓度有关。例如在重水溶液中生长DKDP晶体时,氘在晶体中的浓度$x$(以摩尔分数表示,$x=D/(D+H)$),取决于氘在溶液中的浓度(以摩尔分数$y$表示),其有效分配系数($k_\text{eff}(D) = x/y$)和$y$有关,关系式为
\begin{equation}
k_\text{eff}(D) = 0.68\exp(0.382y),
\end{equation}
D和H在溶液和晶体中的分布是均匀的,在氘化程度较高的溶液中,也可将D$_2$O看成溶剂,而把H$_2$O看成第三个组分(杂质)来计算氢的有效分配系数$k_\text{eff}(H)$
\begin{equation}
k_\text{eff}(H)=(1-x)/(1-y).
\end{equation} %分配系数
\subsection{相稳定区和亚稳相生长}
许多物质在水溶液中可以形成数种不同的晶相。例如硫酸钠在水溶液中形成$\rm Na_2SO_4$和$\rm Na_2SO_4\cdot 10H_2O$两种固相(图3.14)。

\begin{figure}[h]
 \centering
 \includegraphics[width=0.8\textwidth]{fig/cp03/img3.14.jpg}
 \caption{$\rm Na_2SO_4-H_2O$体系。}
\end{figure}

\begin{figure}[h]
 \centering
 \includegraphics[width=0.8\textwidth]{fig/cp03/img3.15.jpg}
 \caption{$\rm EDT-H_2O$体系。}
\end{figure}

EDT在水溶液中,有EDT和$\rm EDT\cdot H_2O$两种结晶(图3.15)。这两个体系都是二组分体系,根据相律,在三相共存时只有一个自由度。如果压力固定(1atm),则体系是不变的。这就是说,两种晶相同时与溶液达成平衡的温度(即相平衡转变温度,以下简称转变温度)是完全确定的。硫酸钠的转变温度为32.4℃,EDT的转变温度为40.6℃。DKDP(通常为$\rm KD_{2x}H_{2(1-x)}PO_4$)在重水溶液中会出现四方和单斜两种结晶相,该体系可看成是$\rm DKDP-D_2O/D_2O+H_2O$准二组分体系,在三相达成平衡时,若压力固定(1atm),则两固相与溶液达成平衡的转变温度随$\rm DKDP-D_2O/D_2O+H_2O$的变化而变化。(图3.16)。对于D2O含量为99.8\%的溶液,DKDP的转变温度为21℃(见图3.11)

\begin{figure}[h]
 \centering
 \includegraphics[width=0.8\textwidth]{fig/cp03/img3.16.jpg}
 \caption{四方和单斜DKDP的稳定区。}
\end{figure}

当溶液温度偏离转变点时,只能存在一种晶相与溶液平衡,另一种晶相将被溶解。但是晶体生长过程是一个不平衡的过程。如果溶液状态点处于两相溶解度曲线以上,如图3.15中的F点,则溶液对EDT无水物和水合物都是过饱和的,两相都可以生长,但无水物是稳定相,水合物是亚稳相,溶液相对于前者过饱和度较大,故生长速度也快。如果状态点在E,则EDT水合物生长,无水物溶解。但如果在无水物的稳定区内引入水合物籽晶,则水合物晶体即可在该区内实现亚稳相生长,长成$\rm EDT\cdot H_2O$大晶体,前提条件是溶液的过饱和度能满足$\rm EDT\cdot H_2O$晶体生长而又不致引起EDT稳定相自发成核。这是完全可能的,因为自发成核所需的过饱和度要比晶种上生长所需的过饱和度大得多。亚稳相生长区的界限可通过实验求出。不同物质和不同相的亚稳相生长区是不一样的。例如图3.14中,转变点以上,$\rm EDT\cdot H_2O$亚稳相生长区较窄(约6℃),而在转变点以下,EDT亚稳相生长区就较宽(约10℃)。对于DKDP,四方和单斜两相溶解度十分接近,亚稳相生长区比较宽,如对转变点为21℃的99.8\%的重水溶液,可以从43℃开始,用降温法成功地培养四方亚稳相晶体。图3.16粗略地示出在不同含氘量的重水溶液中,四方和单斜DKDP的亚稳相生长区。

溶液中的亚稳相晶体溶解度较大,在热力学上是不稳定的,它有自发地转变为溶解度较小的稳定相的趋势。例如,在离转变点较远的单斜稳定区中生长四方DKDP晶体时,在四方晶种上很容易长出单斜晶体,后者一旦出现即迅速生长,使四方母晶迅速被单斜晶体所“蚕食”。同样,在四方稳定区内生长单斜晶体时,也会出现四方晶体“蚕食”单斜母体的情况。

处于热力学不稳定状态的体系,在向其稳定状态过渡时,有时不是过渡到最稳定的状态(即自由能最低的状态),而是过渡到其自由能降低较小的中间状态。这一现象称为Ostwald阶段定律。例如在$\rm Na_2SO_4-H2O$体系中,有时从硫酸钠过饱和溶液中析出的不是稳定的$\rm Na_2SO_4\cdot 10H_2O$,而是亚稳相$\rm Na_2SO_4\cdot 7H_2O$(图3.14)。在DKDP的单斜稳定区中,有时也会析出亚稳的四方DKDP晶体(自发结晶实验证明,在图3.16上下两条虚线间析出的自发结晶既有单斜相,也有四方相),使溶液状态停留在亚稳四方相溶解度曲线附近,而不是直接降到单斜相溶解度曲线附近,虽然后者自由能降低更多,更为稳定。这样就有可能把溶液保持在中间状态,在这样的过饱和溶液中引入亚稳相籽晶,常常可以达到生长亚稳相晶体的目的。

在亚稳的条件下,生长亚稳相晶体有许多优点。许多高温变体如能在较低的温度下作为亚稳相自溶液中生长,则具有在本章开始时所述的优越性。在另一些情况下,由于某一相稳定区的局限性,需要在另一相的稳定区内生长。例如在氘化程度较高的重水溶液中,DKDP的转变温度很低,为了能在通常的温度区间用降温法生长出高含氘量的四方DKDP晶体,也可以在起始温度适当高于转变温度的情况下,使该晶体在单斜相稳定区内生长。图3.17示出的四方DKDP大晶体就是在这样的条件下生长出来的。在育晶器底部可以清楚地看见在生长过程中自发形成的单斜杂晶,由于溶液对这两相都是过饱和的,所以四方相和单斜相能同时存在和生长(但溶液和这两相并不处于平衡状态)。由此可见,亚稳相晶体生长的研究,在理论和实践上都是很有意义的。

\begin{figure}[h]
 \centering
 \includegraphics[width=0.6\textwidth]{fig/cp03/img3.17.jpg}
 \caption{在单斜相稳定区内生长的DKDP大晶体($\rm 5.7\times 5.7\times 4.5cm$)}
\end{figure}

 %相稳定区和亚稳相生长
\setcounter{section}{2}
\section{从溶液中生长晶体的方法}
从溶液中生长晶体过程的最关键因素是控制溶液的过饱和度。使溶液达到过饱和状态,在晶体生长过程中维持其过饱和度的途径有:
\begin{enumerate}[(1)]\itemsep -0.5ex
\item 据溶解度曲线,改变温度。
\item 采取各种方式(如蒸发、电解)移去溶剂,改变溶液成分。
\item 通过化学反应来控制过饱和度。由于化学反应速度和晶体生长速度差别很大,做到这一点是很困难的,需要采取一些特殊的方式,如通过凝胶扩散使反应缓慢进行等。
\item 用亚稳相来控制过饱和度,即利用某些物质的稳定相和亚稳相的溶解度差别,控制一定的温度,使亚稳相不断溶解,稳定相不断生长。
\end{enumerate}

根据晶体的溶解度和温度系数,从溶液中生长晶体的具体方法有下述几种。

\subsection{降温法}
降温法是从溶液中培养晶体的一种最常用的方法。这种方法适用于溶解度和温度系数都较大的物质,并需要一定的温度区间。这一温度区间也是有限制的:温度上限由于蒸发量大而不宜过高,当温度下限太低时,对晶体生长也不利。一般来说,比较合适的起始温度是50---60℃,降温区间以15---20℃为宜。

降温法的基本原理是利用物质较大的正溶解度温度系数,在晶体生长过程中逐渐降低温度,使析出的物质不断在晶体上生长。用这种方法生长的物质的溶解度温度系数最好不低于1.5g/1000g溶液$\cdot$℃,表3.7列出符合此要求的一些物质的数据。

(表3.7  40℃时,一些物质的溶解度及其温度系数)

降温法生长晶体的几种装置,如图3.18、图3.19和图3.20所示。

在降温法生长晶体的整个过程中,必须严格控制温度,并按一定程序降温。研究表明,微小的温度波动就足以在生长的晶体中造成某些不均匀区域。为提高晶体生长的完整性,要求控温精度尽可能高(目前已达±0.001℃),此外还需造成适合晶体生长的其他条件。

(图3.18  水浴育晶装置)

在利用降温法来生长晶体的过程中可不用再补充溶液或溶质。因此,整个育晶器在生长过程中必须严格密封,以防溶剂蒸发和外界的污染。例如从重水溶液中生长晶体时,如果密封不好,溶液中的D$_2$O就会同大气中的H$_2$O汽发生同位素交换而使溶液氘化程度下降。为增加温度的稳定性,育晶器的容量必须要大些(大型育晶器有几十升至上千升),加热保温方式有水浴槽或内部加热、外壳加保温套等。育晶器顶部经常有保持冷凝水回流,底部有电炉加热为好,使得溶液表面和底部都有不饱和层保护,避免自发晶核形成。

育晶器内的控温精度除了与育晶装置的结构有关外,很大程度上取决于控温装置,继电器开关式控温一般可以满足水溶液晶体生长的要求,温度波动可以控制在±0.05℃以下,用PID控温方式,特别是使用可编程温度控制器控温,实现控温精度至±0.005℃并不困难,这对培养高完整性单晶十分有利。

(图3.19 图3.20)

为使溶液温度均匀,并使生长中的各个晶面在过饱和溶液中能得到均匀的溶质供应,要求晶体对溶液作相对运动。这种运动可采取多种形式,如晃动法(晶体固定不动,摇晃整个育晶器)、转晶法(晶体在溶液中作自转,公转或行星式转动)等,其中以晶体在溶液中自转或公转最为常用。为了克服这种方式所造成的的某些晶面迎液而动和使另一些晶面总是背向液流的缺点,转动需要定时换向,即用以下程序进行控制:正转---停转---反转---停转---正转。

降温法控制晶体生长的主要关键是掌握合适的降温速度。使溶液始终处于亚稳区内,并维持适宜的过饱和度,降温速度一般取决于以下述几个因素:
\begin{enumerate}[(1)]
\item 晶体的最大透明生长速度(即在一定条件下不产生宏观缺陷的最大生长速度)。这一数值对不同晶体是有明显差别的(和亚稳区大小有关),例如对硝酸钠(NaNO$_3$)晶体为1mm/d,酒石酸钾钠(KNT)晶体则可达5mm/d以上。对同一晶体该数值还和晶体尺寸有关。
\item 溶解度的温度系数。溶解度的温度系数不但随不同物质而异,而且对同一种物质在不同的温度区间也一样。例如KDP的溶解度曲线在高温部分(50---70℃)温度系数较大,在低温部分(30℃以下)则较小。
\item 溶液的体积V(ml)和晶体生长表面积S(cm$^2$)之比,简称体面比。有些晶体(如KDP型晶体)在生长过程中,生长面积基本不变,而有些晶体(如KNT)在各个方向上都生长,S在生长过程中则在不断增加。
\end{enumerate}

总之,上述三个因素对于不同晶体是有明显差别的;对同一种晶体,这些因素在生长过程中也是在变化的。因此必须从实际出发,对不同的晶体在不同的阶段制定不同的降温程序。一般来说,在生长初期降温要慢,到了生长后期可稍快些。掌握规律后,可按程序实行自动降温。

必须指出,无宏观缺陷的晶体不一定是高质量的晶体(见3.4.3节)。培养用于光学目的高完整性的单晶,其生长速度应当控制的更小一些。

在控制降温过程中,最好能随时测定溶液的过饱和度(见3.1.4节)。同时,一些晶体生长现象[如生长涡流的强弱,晶面相对大小的变化,次要面的出现和消失,晶面花纹(见3.4.3节)等]往往是溶液过饱和度偏高或偏低以及晶体均匀性将遭破坏的“信号”。这些现象也可作为估计过饱和度、控制降温速度的参考信号。 %降温法
\subsection{流动法(温差法)}
流动法生长晶体装置一般由三部分组成(图3.21):生长槽(育晶器)C,溶解槽A和过热槽B。三槽之间的温度是B槽的温度高于A槽的温度高于A槽的,而A槽的温度又高于C槽的,A槽中过剩的原料在不断地搅拌下溶解,使溶液在高于C槽的温度在饱和,然后经过滤器进入过热槽B。过热槽的温度一般高于生长槽温度5--10℃。可以充分溶解从溶解槽流入的微晶,以提高溶液的稳定性。经过过热后的溶液用泵打入生长槽C,C槽的溶液是过饱和的,保证晶体生长有一定的驱动力。由于晶体的生长,从而使变稀的溶液流到溶解槽溶解原料,使溶液重新达到饱和,溶液如此循环流动,晶体不断生长。晶体的生长速度受到溶液的流动速度和A,C两槽温差的控制。这种方法的优点是生长温度和过饱和度都固定。使晶体始终在最有利的温度和最合适的过饱和度下生长,避免了因生长温度和过饱和度变化而产生的杂质分凝不均匀和生长带等缺陷,使晶体完整性更好。流动发的另一个优点就是生长大批量的晶体和培养大单净不收溶解度和溶液体积的影响,只受生长容器大小的限制。日本大阪大学用这种方法长出400$\times$400$\times$600mm大KDP晶体,生长速度达2mm/d。流动法的缺点是设备比较复杂,调节三槽之间适当的温度梯度和溶液流速之间的关系需要有一定的经验。
\begin{figure}[h]
 \centering
 \includegraphics[width=0.6\textwidth]{fig/cp03/img3.21.jpg}
 \caption{循环流动育晶装置}
\end{figure}

采用适当装置也可以利用浓差自然对流来生长晶体。图3.22示出利用亚稳相和稳定相溶解度的差别通过浓差对流来生长$\alpha$-LiIO$_3$体的装置。两连通的玻璃槽,右边装$\beta$-LiIO$_3$原料,左边为生长槽,由于在20--30℃时,$\beta$-LiIO$_3$的溶解度比$\alpha$-LiIO$_3$的大1--2\%。浓度较大的LiIO$_3$溶液靠自然对流进入左边生长区,生长槽下部设置加热器,将溶液温度保持在40℃,造成对$\alpha$-LiIO$_3$的过饱和,析出的溶质在$\alpha$-LiIO$_3$种子上生长,变稀的溶液上升流回右边的原料槽重新溶解$\beta$-LiIO$_3$,原料槽靠空气冷却稳定在20--30℃。
\begin{figure}[h]
 \centering
 \includegraphics[width=0.6\textwidth]{fig/cp03/img3.22.jpg}
 \caption{浓差对流法生长LiIO$_3$晶体的装置}
\end{figure} %浓差法
\subsection{蒸发法}
蒸发法生长晶体的基本原理是将溶剂不断蒸发移去,而使溶液保持在过饱和状态,从而使晶体不断生长,这种方法比较适合于溶解度较大而其温度系数很小或是具有负温度系数的物质(表3.8)。蒸发法和流动法一样,晶体生长也是在恒温下进行的。不同是流动法用补充溶质,而蒸发法用移去溶剂来造成过饱和度。

蒸发法生长晶体的装置和降温法的装置十分类似。所不同的是在降温法中,育晶器中蒸发产生的冷凝水全部回流,而蒸发法则是部分回流。降温法通过降温速度来控制过饱和度,而蒸发法则是通过控制回流比(蒸发量)来控制过饱和度的。

(表3.8  一些适用于蒸发法….及其温度系数)

蒸发法生长晶体的装置有许多类型。图3.23示出的是比较简单的一种:在严格密封的育晶器上方设置冷凝器(可通水冷却)。溶剂自溶液表面不断蒸发,水蒸汽在冷凝器上凝结,并积聚在其下方的小杯内,再用虹吸管按控制量移出育晶器外,在晶体生长过程中,取水速度应小于冷凝速度,使大部分冷凝水(包括器壁上的)回流到液面上去,否则液面上易产生自发结晶。这种装置比较适合在较高的生长温度($>$60℃)使用。温度较低。蒸发量太小,不能满足晶体生长的需要。

(图3.25 蒸发法育晶装置和图3.24 生长NdPP晶体的装置)

有时体系中某一成分(如水)的蒸发并不是作为溶剂蒸发直接导致晶体生长,而是引起化学反应,间接导致晶体生长,例如在$\rm Nd_2O_3-H_3PO_4$(或$\rm Nd_2O_3-P_2O_5-H_2O$)体系中生长五磷酸钕($\rm NdP_5O_{14}$简称NdPP)晶体,其形成机制可能是
$$ \rm 14\ H_3PO_4 + Nd_2O_3 \xrightarrow{>260℃} 2\ NdP_5O_{14}+2\ H_4P_2O_7+17\ H_2O\uparrow$$

NdPP在焦磷酸($\rm H_4P_2O_7$)中有较大的溶解度,所以不会从溶液中析出,当温度升至300℃以上,焦磷酸逐渐脱水,形成多聚偏磷酸,NdPP在其中溶解度很小,在升温和蒸发过程中,由于焦磷酸浓度降低而使NdPP在溶液中到达过饱和而结晶出来
$$\rm n\ H_4P_2O_7 + NdP_5O_{14} \xrightarrow{>300℃} 2\ (HPO_3)_n + NdP_5O_{14}\downarrow + n\ H_2O\uparrow $$

据此机理,采用图3.24所示装置,在一定的温度下,控制水的蒸发速率就可以生长出质量较好的NdPP晶体。

这种晶体生长方式实际上是晶体在无机溶剂(焦磷酸)的溶液中,通过焦磷酸脱水蒸发而产生缩聚反应,使溶剂不断减少,并使溶质(NdPP)从其饱和溶液中结晶出来的过程。因此将其归入蒸发法。 %蒸发法
\subsection{电解溶剂法}
电解溶剂法是从溶液中生长晶体的一种独特的方法。其原理基于用电解法分解溶剂,以除去溶剂,使溶液处于过饱和状态。显然这种方法只能应用于溶剂可以被电解而其产物很容易自溶液中移去(如气体)的体系。同时还要求所培养的晶体在溶液中能导电而又不被电解。因此,这种方法特别适用于一些稳定的离子晶体的水溶液体系。

电解溶剂法的一半装置如图3.25所示。育晶器中装有铂电极,也起电解槽的作用,当通以稳定的直流电,溶剂就被电解,其速度由电流密度控制。溶液要搅拌以免产生浓差极化。溶液表面用流动液层(如邻二甲苯)覆盖以防溶剂蒸发,电解的产物从冷凝器中排除,在生长过程中,溶液pH应保持稳定。

\begin{figure}[h]
 \centering
 \includegraphics[width=0.8\textwidth]{fig/cp03/img3.25.jpg}
 \caption{电解溶剂法生长晶体装置图。}
\end{figure}

和流动法和蒸发法一样,用电解溶剂法生长晶体也是恒温下进行的。由于过饱和度使用直流电准确控制的,因此和生长温度关系不大,也可在室温下进行(这一点比蒸发法优越,因为温度较低时,蒸发量小,难以控制),也不需要知道溶解度曲线的情况。所以这种方法既适用于溶解度温度系数比较小的晶体,也适用于生长有数种晶相存在,而每种晶相仅在一定温度范围内才能稳定存在的晶体。

用电解溶剂法来生长KDP型(特别是DKDP晶体)获得了满意的结果。因为溶液中存在的常导致这些晶体柱面楔化的一些金属杂质离子(如Fe$^{3+}$,Al$^{3+}$,Cr$^{3+}$等)可以在电解过程中除去,从而消除这些杂质的有害影响。对DKDP晶体可以在低于其转变点的温度下生长,以防止单斜相的干扰,由于分解H$_2$O所需的能量比分解D$_2$O的要低,溶液中的H$_2$O在电解过程中比较容易除去。溶液在生长过程中可以保持较高的氘化程度。图3.26示出在普通转晶育晶器(图3.18)基础上改装的电解溶剂法生长这一类晶体的装置。阳极置于育晶器底部,为使电流均匀通过溶液,在溶液上方安置了两个电极,阴极放在喇叭形口的塑料管内,口上用尼龙网覆盖,使在阴极上产生的氢气引入通风良好的空间,防止其重新进入溶液。在阳极上产生的氧气进入育晶器中溶液上方的空间,保持正压,它投注于减少大气中的水汽和溶液中的氘发生交换而降低其含氘量。由于电解交流也会产生热量,因此也可以不使用底部加热器,而是将交流电直接通过电极加热,使用交、直流并用的加热—电解联合控制装置。晶体在溶液中转动以使溶液浓度均匀。由于溶液是强缓冲溶液,所以电解过程中溶液pH变化不大。

\begin{figure}[h]
 \centering
 \includegraphics[width=0.8\textwidth]{fig/cp03/img3.26.jpg}
 \caption{由通常转晶育晶器改装的电解溶剂育晶装置。}
\end{figure}

对于从重水溶液中生长高质量的KDP型晶体,电解溶剂法是一种有前途的方法。

除了上述四种从水溶液中生长晶体的方法外,还有凝胶法,有机溶剂法等,这些内容将在下面独立的章节中进行论述。

上述的从溶液中生长晶体的各种方法中,以降温法、蒸发法、流动法最为常用,大部分水溶性晶体都是用这些方法培养的。表3.9列出了从溶液中生长的一些晶体的单晶培养和一般的结晶方法。

(表3.9  从溶液中生长的一些重要晶体)

 %电解溶剂法

