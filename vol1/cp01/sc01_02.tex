\subsection{力学平衡}
若两相都为流体时,我们将它们放置在一个恒温恒容箱内。根据自由能判据,得出平衡状态时应有
$$dF=dF_A+dF_B=0.$$
设两相之间无物质交换,但可改变体积,则
\begin{equation}
\begin{aligned}
dF_A&=-S_AdT_A-P_AdV_A, \\
dF_B&=-S_BdT_B-P_BdV_B.
\end{aligned}
\end{equation}
由于恒温,故$dT_A=dT_B=0$;由于总体积不变(恒容),故$dV=dV_A+dV_B=0$,得$dV_A=-dV_B$,于是$dF=dF_A+dF_B=\left(P_A-P_B\right)dV_A=0$。但$dV_A\neq 0$,故
\begin{equation}
P_A=P_B.
\end{equation}
这就是说,当两相在恒温,且总体积不变(恒容)的情况下处于平衡状态时,两相的压强应相等,这里假设两相间的接触面是平面,如果接触面是弯曲界面(如图1.2所示),且$A$相是球体时,则应加入表面能项,即$dF_A$中除了$P_AdV_A$外,还有$\sigma d\alpha_A$项,$\sigma$表示比表面能,$\alpha_A$表示$A$相的表面积。

(图1.2)

\noindent 于是
\begin{align*}
dF_A&=-S_AdT_A-P_AdV_A+\sigma d\alpha_A,\\
dF_B&=-S_BdT_B-P_BdV_B,
\end{align*}
将$dV_B=-dV_A$代入,则两式相减后,得
$$-(P_A-P_B)dV_A+\sigma d\alpha_A=0$$
故
$$P_A-P_B=\frac{\sigma d\alpha_A}{dV_A}.$$
由于球体的$\dfrac{d\alpha_A}{dV_A}=\dfrac{2}{R}$,代入后,得
\begin{equation}
P_A-P_B=\frac{2\sigma}{R}
\end{equation}
当$R\rightarrow\infty$,即曲面变为平面时,式(1.6)即变为式(1.5)。