\subsection{传质平衡}
考虑由$A,B$两相所组成的体系,处在恒温恒压下,达到力学平衡和热平衡(即$T_A=T_B=T,P_A=P_B=P$)。设有$dn_{iB}$摩尔的组分$i$由$B$相进入$A$相内。根据热力学中开启系自由能表达式,整个体系的亥姆霍兹自由能改变量(因$dP=0,dG=DF+PdV=\sum\limits_i{\mu_idn_i}$,故可用$dF=-PdV+\sum\limits_i{\mu_idn_i}$表示)
\begin{equation}
dF=dF_A+dF_B=-PdV_A-PdV_B+(\mu_{iB}-\mu_{iA})dn_{iB},
\end{equation}
式中利用了$dn_{iA}=-dn_{iB}$,$\mu_{iB}$是第$i$组分在$B$相中的化学势,$\mu_{iA}$是该组分在$A$相中的化学势,而前两项是对体系所做\footnote{原文作“作”。}的功,设为$w$,故有
\begin{equation}
dF=w+(\mu_{iB}-\mu_{iA})dn_{iB}.
\end{equation}
根据自由能平衡判据,$dF\leq w$,则上式可谢伟
\begin{equation}
(\mu_{iB}-\mu_{iA})dn_{iB}\leq 0.
\end{equation}
若过程为不可逆时,则取$(\mu_{iB}-\mu_{iA})dn_{iB}<0$,此式说明$(\mu_{iB}-\mu_{iA})$的符号与$dn_{iB}$的符号相反。若$dn_{iB}$为负(即物质由$B$相进入$A$相)时,则有$\mu_{iB}-\mu_{iA}>0,\mu_{iB}>\mu_{iA}$。若过程为可逆时,则有$(\mu_{iB}-\mu_{iA})dn_{iB}=0$,但$dn_{iB}\neq 0$,故有
\begin{equation}
\mu_{iB}-\mu_{iA}=0\text{ 或 }\mu_{iB}=\mu_{iA}.
\end{equation}
由此得出结论,物质由化学势较高的相进入化学势较低的相。达到平衡时,该物质在两相中的化学势相等。

同样,对其他组分也可得出类似上述的结果,同时也可推广至两个相以上的体系。所以在由若干个相组成的体系中,当该体系达到平衡状态时,则组成体系的每一组份在各个相内的化学势彼此相等, 即
\begin{equation}
\mu_{iA}=\mu_{iB}=\mu_{iC}=\cdots=\mu_{iN}
\end{equation}
等。因此,在恒温恒压下,当体系达到平衡时,各组分在整个体系各个相之内的化学势必须彼此相等,否则,在各个相之间将存在物质的传递过程,物质将从化学势高的相转移到化学势低的相,直到组分在各相内的化学势相等为止。
