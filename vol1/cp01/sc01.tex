\section{相平衡及相变}
所谓相,是指体系中均匀一致的部分,它与别的部分有明显的分界线。例如,在大气压力下,冰与水混合共存时,冰本身是均匀一致的,它与水总有一定的分界面。不管是一大块冰,还是许多小冰块,冰所表现的物理状态和和化学状态总是一样的。所以在这个体系中,冰是一个相,水是另一个相。有时在同一种物质的固体状态(例如冰)下,由于温度或/和压强的不同,也可能有不同的结构(对称性)存在。这时候同时一种物质在固态下的不同结构,我们认为也是属于不同的相。但是在气体状态时,无论是由单一种气体或是几种不同气体组成,都是单一个相。

两相处于平衡状态时,其宏观性质有什么关系呢?我们可以根据热力学中的平衡判据来求出。

\subsection{热平衡}
设将$A$和$B$两个相封闭在一个与环境隔绝(与环境无热量和物质的交换)的体系内,$A$与$B$两相之间只有热量交换,即$A,B$两相间的隔板完全固定,只能导热。如图1.1所示,设此时从$A$有微量的热传到$B$内,则$A,B$两相的内能变化为
\begin{equation}
\begin{aligned}
\mathrm{d}U_A-T_AdS_A-P_AdV_A, \\
\mathrm{d}U_B-T_BdS_B-P_AdV_B
\end{aligned}
\end{equation}